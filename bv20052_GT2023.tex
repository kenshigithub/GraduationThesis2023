% 総合研究論文サンプルファイル
%
% ・和文題名, 英文題名, 学籍番号, 姓, 名, 指導教員名, 職位 を記入する
%   (====== で囲まれた範囲).
% ・顔写真 (履歴書の余りなど) を用意し, 所定の位置に貼る.
%   デジタルデータ (JPG 等) があれば, それを使ってもよい (下記設定変更).
%
\documentclass[a4j,12pt]{jarticle}
\usepackage{amsmath}
\usepackage{amssymb}
\usepackage[dvipdfmx]{graphicx}%画像挿入
%\usepackage{theorem}%定義や定理の環境
\usepackage{url}
%必要なパッケージを追加
%定理環境の設定

\usepackage{amsthm}
\theoremstyle{definition}
\newtheorem{theorem}{定理}[section]
\newtheorem{definition}[theorem]{定義}
\newtheorem{lemma}[theorem]{補題}
\newtheorem{axiom}[theorem]{公理}
\newtheorem{proposition}[theorem]{命題}
\newtheorem{corollary}[theorem]{系}
\newtheorem{example}[theorem]{例}
\newtheorem*{theorem*}{定理}
\newtheorem*{definition*}{定義}
\newtheorem*{lemma*}{補題}
\newtheorem*{axiom*}{公理}
\newtheorem*{proposition*}{命題}
\newtheorem*{corollary*}{系}
\newtheorem*{example*}{例}
\renewcommand\proofname{\bf 証明}

%スタイルファイルを追加したい場合は以下のように書く
\usepackage{graphicx}
\usepackage{amsmath,amssymb}
%\usepackage{eclbkbox}
\usepackage{tikz}

%\newcommandでマクロを定義できる
\newcommand{\macrotest}{\LaTeX のマクロは便利}
\newcommand{\PD}[2]{\frac{\partial {#1}}{\partial {#2}}}%引数もとれる
\newcommand{\combination}[2]{{}_{#1} \mathrm{C}_{#2}}



%========================================
% 各自の内容を記述
%
\def\YEAR{2023}  % 総合研究着手年度
\def\TITLE{多様体の次元を調べる方法}  % 和文題名
\def\ETITLE{Methods for Investigating the Dimension of Manifolds}  % 英文題名
\def\NUMBER{BV20052}  % 学籍番号
\def\FAMILYNAME{\ruby{青見}{あおみ}}  % 姓
\def\FIRSTNAME{\ruby{健志}{けんし}}  % 名
\def\ADVISER{亀子 正樹}  % 指導教員 氏名
\def\POSITION{教授}  % 指導教員 職位
\def\JPG{SIT.png}  % 顔写真ファイル名 (なければ空白で可)
%
% 顔写真の設定: 下のいずれかの行を活かし, 他方をコメントアウト
%
\def\FACE{\WAKU}   % 枠線のみ表示
%\def\FACE{\IMAGE}  % 顔写真ファイル使用時
%========================================
% ルビの定義
%
\def\ruby#1#2{%
\leavevmode
\setbox0=\hbox{#1}\setbox1=\hbox{\scriptsize#2}%
\ifdim\wd0>\wd1 \dimen0=\wd0 \else \dimen0=\wd1 \fi
\hbox{\kanjiskip=\fill
 \vbox{\hbox to \dimen0{\scriptsize \hfil#2\hfil}%
 \nointerlineskip
 \hbox to \dimen0{\hfil#1\hfil}}}}
%
% 写真の定義
\def\WAKU{\framebox[30mm]{\rule{0mm}{38mm}\raisebox{18mm}{\smash{
 \parbox{25mm}{\begin{center}顔写真\\ 横3cm\\ ×\\ 縦4cm\end{center}}}}}
 \\[7mm]}
\def\IMAGE{\parbox{33mm}{\rule{0mm}{38mm}\raisebox{18mm}{\smash{
 \parbox{31mm}{\includegravarphics[height=40mm,keepaspectratio]{\JPG}}}}}
 \\[12mm]}
%----------------------------------------
\begin{document}
\thispagestyle{empty}
\begin{center}
 \YEAR{}年度 \\
 芝浦工業大学\quad システム理工学部 \\
 数理科学科 \\[12mm]
 \huge 総合研究論文
\end{center}
\vspace{16mm}\par
%
% 題名
%
\noindent\smash{
 \begin{minipage}[t]{.98\linewidth}
  \begin{center}
   \Huge\bf\TITLE  % 和文題名
   \\[1ex]
   \Large\bf\ETITLE  % 英文題名
  \end{center}
 \end{minipage}}
\vspace{47mm}\par
%
% 顔写真, 学籍番号, 氏名, 指導教員
%
\begin{center}
 \FACE                                        % 顔写真
 \Large \NUMBER                     \\[7mm]   % 学籍番号
 \huge  \FAMILYNAME\quad \FIRSTNAME \\[15mm]  % 氏名
 \Large 指導教員: \ADVISER ~ \POSITION
\end{center}
\vfill
\newpage
\setcounter{page}{1}
\tableofcontents	%目次
\thispagestyle{empty}
\newpage
%%%%%%↓ここから本文↓%%%%%%
\section{はじめに}
\newpage
%
\section{準備}
%\subsection{m次元数空間}
%%\subsection{ベクトル空間}
\subsection{連続写像と$C^r$級写像}
\subsection{位相空間}
\newpage
%
\section{$C^r$級多様体と$C^r$級写像}
\subsection{$C^r$級多様体}
\begin{definition}
    位相空間$X$の開集合$U$から$m$次元数空間$\mathbb{R}^m$
    のある開集合$U'$への同相写像
    $$\varphi:U\rightarrow U'$$
    があるとき, $(U, \varphi)$を$m$次元座標近傍といい, 
    $\varphi$を$U$上の局所座標系という. 
\end{definition}
$(U,\varphi)$には局所座標系$(x_1, \cdots ,x_m)$
    が描かれていると考え, $(U;x_1, \cdots ,x_m)$
    とも表すこととする. 
\begin{definition}
    $(U, \varphi)$を位相空間$X$内の局所座標近傍とする.
    $U$内の任意の点$p$に対して$\varphi(p) \in \mathbb{R}^m$
    であるから, 
    $$\varphi(p)=(x_1, \cdots ,x_m)$$
    と書ける.$(x_1, \cdots ,x_m)$を$(U, \varphi)$に関する$p$
    の局所座標という.
\end{definition}
\begin{definition}
    位相空間$M$が次の条件(1),(2)を満たすとき, 
    $M$を$m$次元位相多様体という. 
    \begin{itemize}
        \item[(1)]$M$はハウスドルフ空間である.
        \item[(2)]$M$内の任意の点$p$に対して, 
        $p$を含む$m$次元座標近傍$(U,\varphi)$が存在する.
    \end{itemize}
\end{definition}
\begin{definition}
    $m$次元位相多様体$M$の$2$つの座標近傍$(U, \varphi)$, 
    $(V, \psi)$が交わっているとき, 同相写像
    $$\psi \circ \varphi:\varphi(U\cap V)\rightarrow \psi(U\cap V)$$
    を$(U, \varphi)$から$(V, \psi)$への座標変換という. 
\end{definition}
\begin{definition}\label{def:C^r manifold}
    $r\geq 1$を自然数または$\infty$とする. 
    位相空間$M$が次の条件(1), (2), (3)を満たすとき, 
    $M$を$m$次元$C^r$級多様体という.
    \begin{itemize}
        \item[(1)]$M$はハウスドルフ空間である.
        \item[(2)]$M$は$m$次元座標近傍により被覆される. 
        すなわち, $M$の$m$次元座標近傍からなる族
        $\{(U_\alpha, \varphi_\alpha)\}_{\alpha \in A}$
        があって, 
        $$M = \bigcup_{\alpha \in A}U_\alpha$$
        が成り立つ. 
        \item[(3)]$U_\alpha \cap U_\beta \neq \phi$
        であるような任意の$\alpha$, $\beta$に対して, 座標変換
        $$\psi \circ \varphi:\varphi(U\cap V)\rightarrow \psi(U\cap V)$$
        は$C^r$級写像である. 
    \end{itemize}
\end{definition}
\begin{theorem}
    $m$次元球面$S^m \in \mathbb{R}^{m+1}$を
    $$S^m=\{(x_1,\cdots x_{m+1})|x_1^2+\cdots +x_{m+1}^2=1\}$$
    と定義すると, $S^m$は$m$次元$C^{\infty}$級多様体である. 
\end{theorem}
\begin{proof}
    定義\ref{def:C^r manifold}の条件(1), (2), (3)を確かめる. 
    \begin{itemize}
        \item[(1)]$\mathbb{R}^{m+1}$はハウスドルフ空間
        であるから, その部分空間として, $S^m$は
        ハウスドルフ空間である. 
        \item[(2)]$S^m$の$2(m+1)$個の開集合
        $U_i^+$, $U_i^-$ $(i=1,\cdots ,m+1)$を
        次のように定義する. 
        $$U_i^+ = \{(x_1, \cdots x_i, \cdots ,x_{m+1})\in S^m|x_i>0\}$$
        $$U_i^- = \{(x_1, \cdots x_i, \cdots ,x_{m+1})\in S^m|x_i<0\}$$
        $S^m$はこれら$U_i^+$, $U_i^-$ $(i=1,\cdots ,m+1)$
        で被覆される. 写像$\varphi_i^+:U_i^+ \rightarrow \mathbb{R}^m$, 
        $\varphi_i^-:U_i^- \rightarrow \mathbb{R}^m$を
        それぞれ次のように定義する. 
        $$\varphi_i^+(x_1,\cdots ,x_i,\cdots, x_{m+1})=(x_1,\cdots ,\hat{x_i},\cdots ,x_{m+1})$$
        $$\varphi_i^-(x_1,\cdots ,x_i,\cdots, x_{m+1})=(x_1,\cdots ,\hat{x_i},\cdots ,x_{m+1})$$
        ここで, $\hat{x_i}$は$x_i$を取り去るという意味である. このとき, 
        $$(\varphi_i^+)^{-1}(x_1,\cdots ,\hat{x_i},\cdots ,x_{m+1})=(x_1,\cdots ,\sqrt{1-||(x_1,\cdots ,\hat{x_i},\cdots ,x_{m+1})||^2},\cdots ,x_{m+1})$$
        $$(\varphi_i^-)^{-1}(x_1,\cdots ,\hat{x_i},\cdots ,x_{m+1})=(x_1,\cdots ,-\sqrt{1-||(x_1,\cdots ,\hat{x_i},\cdots ,x_{m+1})||^2},\cdots ,x_{m+1})$$
        であり, $\varphi_i^+$, $\varphi_i^-$はそれぞれ, $U_i^+$, 
        $U_i^-$から$\mathring{D}^m$への同相写像である.
        ただし, $\mathring{D}^m$は$\mathbb{R}^m$の原点を
        中心とするm次元単位開円板である. 
        よって, $S^m$は$2(m+1)$個の座標近傍$(U_1^+,\varphi_1^+),
        (U_1^-,\varphi_1^-),\cdots ,(U_{m+1}^+,\varphi_{m+1}^+),(U_{m+1}^-,\varphi_{m+1}^-)$
        で被覆される. 
        \item[(3)]2(m+1)個の座標近傍の間の座標変換がすべて
        $C^{\infty}$級であることを示す. \\
        $1\leq a, b\leq 2(m+1)$を満たす互いに異なる自然数$a$, $b$
        に対して, 
        \begin{itemize}
            \item[(i)]$(U_a^+,\varphi_a^+)$と$(U_b^+,\varphi_b^+)$
            \item[(ii)]$(U_a^-,\varphi_a^-)$と$(U_b^-,\varphi_b^-)$
            \item[(iii)] $(U_a^+,\varphi_a^+)$と$(U_b^-,\varphi_b^-)$ 
        \end{itemize}
        の間の座標変換を調べればよい. \\
        (i)の場合, 
        \begin{eqnarray*}
            &&U_a^+\cap U_b^+ =\{(x_1,\cdots ,x_a,\cdots ,x_b,\cdots ,x_{m+1})\in S^m|x_a>0,\ x_b>0\}\\
            &&\varphi_a^+(U_a^+\cap U_b^+) =\{(x_1,\cdots ,\hat{x_a},\cdots ,x_b,\cdots ,x_{m+1})\in \mathring{D}^m|x_b>0\}\\
            &&\varphi_b^+(U_a^+\cap U_b^+) =\{(x_1,\cdots ,x_a,\cdots ,\hat{x_b},\cdots ,x_{m+1})\in \mathring{D}^m|x_a>0\}\\
            &&(\varphi_a^+)^{-1}x_1,\cdots ,\hat{x_a},\cdots ,x_b,\cdots x_{m+1})\\
            &&=(x_1,\cdots ,\sqrt{1-||(x_1,\cdots ,\hat{x_a},\cdots ,x_b,\cdots x_{m+1})||^2},\cdots ,x_b,\cdots x_{m+1})
        \end{eqnarray*}
        この式から
        \begin{eqnarray*}
        &&\varphi_b^+\circ(\varphi_a^+)^{-1}(x_1,\cdots ,\hat{x_a},\cdots ,x_b,\cdots ,x_{m+1})\\
        &&=(x_1,\cdots ,\sqrt{1-||(x_1,\cdots ,\hat{x_a},\cdots ,x_b,\cdots ,x_{m+1})||^2},\cdots ,\hat{x_b},\cdots ,x_{m+1})
        \end{eqnarray*}
        これは$||(x_1,\cdots ,\hat{x_a},\cdots ,x_b,\cdots ,x_{m+1})||^2<1$の
        範囲で$C^{\infty}$級なので, \\$\varphi_b^+\circ(\varphi_a^+)^{-1}$
        は定義域$\varphi_a^+(U_a^+\cap U_b^+)\subset \mathring{D}^m$
        で$C^{\infty}$級である. \\
        同様にして(ii), (iii)の場合についても座標変換が$C^{\infty}$級である
        こと分かる. \\
    \end{itemize}
    以上より, $S^m$は$m$次元$C^{\infty}$級多様体であることが分かった. 
\end{proof}
\begin{definition}
    $\{(U_\alpha ,\varphi _\alpha)\}_{\alpha \in A}$
    を$m$次元$C^r$級多様体$M$の$1$つの$m$次元$C^r$級
    座標近傍とする. $M$の開集合$V$と$V$から$\mathbb{R}^m$
    への写像$\psi$の対$(V,\psi)$が次の条件(1), (2)
    を満たしているとき, $(V,\psi)$は
    $\{(U_\alpha ,\varphi _\alpha)\}_{\alpha \in A}$
    と両立するという. 
    \begin{itemize}
        \item[(1)]$\psi:V\to \mathbb{R}^m$の像
        $\psi (V)$は$\mathbb{R}^m$の開集合で
        $\psi$は$V$から$\psi (V)$への同相写像
        である. 
        \item[(2)] $V\cap U_\alpha \neq \phi$
        となる$\alpha$に対しては, 
        $\psi \circ \varphi ^{-1}:
        \varphi(V\cap U_\alpha)\to
        \psi(V\cap U_\alpha)$は$C^r$級写像である. 
    \end{itemize}
\begin{proposition}
    $(V_1,\psi_1)$, $(V_1,\psi_1)$がともに与えられた
    座標近傍系$\{(U_\alpha ,
    \varphi _\alpha)\}_{\alpha \in A}$
    と両立していて, $V_1\cap V_2\neq \phi$
    とすると, $\psi_2\circ \psi_1^{-1}$は$C^r$級
    写像となる. 
\end{proposition}
\begin{proof}
    任意の点$x\in V_1\cap V_2$に対し, 
    $x_\in U_\alpha$となる$(U_\alpha, 
    \varphi_\alpha)$が座っ票近傍系の中に
    存在する. 
    $$\psi_2\circ \psi_1^{-1}:
    \psi_1(V_1\cap V_2\cap U_\alpha)\to
    \psi_2(V_1\cap V_2\cap U_\alpha)$$
    は$(\psi_2\circ \varphi_\alpha^{-1})\circ
    (\psi_1\circ \varphi_\alpha^{-1})^{-1}$
    と分解され, 
    $(\psi_2\circ \varphi_\alpha^{-1})$, 
    $(\psi_1\circ \varphi_\alpha^{-1})^{-1}$
    はそれぞれ$x\in V_1\cap V_2$
    に関して$C^r$級であるから, 
    $\psi_2\circ \psi_1^{-1}$も
    $x$に関して$C^r$級である. いま, 
    $x\in V_1\cap V_2$は任意であったから, 
    $\psi_2\circ \psi_1^{-1}$は$C^r$級
    写像である. 
\end{proof}
\begin{definition}
    位相空間$M$上に$C^r$級座標近傍系
    $\mathcal{S}=
    \{(U_\alpha, \varphi_\alpha)\}_{\alpha\in A}$
    が与えられているとする. 
    $\mathcal{S}$
    と両立する$(V,\psi)$の全体集合
    $\mathcal{M}=\mathcal{M}(\mathcal{S})$を
    $M$上の極大な$C^r$級座標近傍系とする. 
    ここで, 極大とは$\mathcal{M}$と両立する
    全ての$(U,\varphi)$が$\mathcal{M}
    =\mathcal{M}(\mathcal{S})$
    に属することをいう. この$\mathcal{M}
    =\mathcal{M}(\mathcal{S})$
    のことを$\mathcal{S}$から決まる
    $M$の$C^r$級極大座標近傍系という. 
\end{definition}
以降, $C^r$級多様体$M$が$C^r$級座標近傍系
$\mathcal{S}$によって定義されたとき, $M$には
$\mathcal{S}$から決まる$C^r$級極大座標近傍系
$\mathcal{M}(\mathcal{S})$を与え, 
$(M,\mathcal{M}(\mathcal{S}))$を
考えるものとする. 
\end{definition}

\subsection{$C^s$級写像}
$M$, $N$を$m$次元, $n$次元の$C^r$級多様体とし, 
$f$を連続写像とする. 
\begin{definition}\label{def:local coordinate display}
$M$, $N$の$C^r$級
座標近傍$(U,\varphi)$, 
$(V,\psi)$が存在し, 
$$f(U)\subset V$$
が成り立つ. 
$U$と$V$の中には座標近傍系$(x_1,\cdots x_m)$, 
$(y_1,\cdots y_n)$があるから, 
$f|U:U\to V$を
$$(y_1,\cdots y_n)=
\psi^{-1} \circ f\circ \varphi(x_1,\cdots x_m)$$
と表示することができる. この表示の
$\psi^{-1} \circ f\circ \varphi$を$f$に書き直した
もの($(y_1,\cdots y_n)=f(x_1,\cdots x_m)$)
を$(U;x_1,\cdots x_m)$と$(V;y_1,\cdots y_n)$
に関する$f$の局所座標表示という. 
\end{definition}
\begin{definition}\label{def:C^s map}
    連続写像$f:M\to N$が$1$点$p\in M$において
    $C^s$級であるとは, $p$を含む$M$の$C^r$級
    座標近傍$(U;x_1,\cdots x_m)$と
    $f(p)$を含む$N$の
    座標近傍$(V;y_1,\cdots y_n)$が存在して, 
    \begin{itemize}
        \item[(1)]$f(U)\subset V$
        \item[(2)]$(U;x_1,\cdots x_m)$と
        $(V;y_1,\cdots y_n)$に関する$f$の
        局所座標表示が$C^s$級である.
        (ただし, $0\leq s \leq r \leq \infty$) 
    \end{itemize}
    この$2$つの条件が成り立つことである. 
    $f$が任意の点$p\in M$で$C^s$級であるとき, 
    $f$は$C^s$級写像であるという. 
\end{definition}

\begin{proposition}
    連続写像$f:M\to N$が$1$点$p\in M$において
    $C^s$級であるという性質は, $p$, $f(p)$を
    それぞれ含む$C^r$級座標近傍$(U,\varphi)$, 
    $(V, \psi)$の選び方によらない. 
    (ただし, $0\leq s \leq r \leq \infty$)
\end{proposition}
\begin{proof}
    $f$は$p\in M$において$(U,\varphi)$, 
    $(V, \psi)$に関して$C^s$級であると仮定する. 
    $f(U')\subset V'$, $U'=U$, $V'=V$となるような
    $p$, $f(p)$をそれぞれ含む別の$C^r$級座標近傍
    $(U',\varphi')$, $(V, \psi')$をとると, 
    $\psi'\circ f\circ \varphi'^{-1}$は
    $$\psi'\circ f\circ \varphi'^{-1}=
    (\psi'\circ \psi^{-1})\circ
    (\psi \circ f\circ \varphi^{-1})\circ
    (\varphi \circ \varphi')$$
    と分解される. $\psi'\circ \psi^{-1}$, 
    $\varphi \circ \varphi'$はそれぞれ$M$, $N$
    における座標変換であるから$C^r$級であり, 
    $\psi \circ f\circ \varphi^{-1}$は
    仮定より$C^s$級であるから, 
    $\psi'\circ f\circ \varphi'^{-1}$は
    $C^s$級である. よって命題は証明された. 
\end{proof}
\begin{proposition}
    $M$, $N$, $Q$を$C^r$級多様体とする. 
    $f:M\to N$, $g:N\to Q$がともに$C^s$級
    写像なら, 合成写像$g\circ f:M\to Q$
    も$C^s$級である. 
    (ただし, $0\leq s \leq r \leq \infty$)
\end{proposition}
\begin{proof}
    \begin{itemize}
        \item[(1)]
        $p$を$M$の任意の点とすると, $g$は$C^s$
        級写像より, $g(V)\subset W$となうような
        $f(p)$, $g(f(p))$を含む座標近傍
        $(V,\psi)$, $(W, \omega)$が存在する. 
        また, $f$は$C^s$
        級写像より, $f(U)\subset V$となうような
        $p$を含む座標近傍$(U,\varphi)$が存在する. 
        よって, 任意の点$p\in M$において
        $g\circ f(p)\subset W$となるような
        $p$, $g\circ f(p)$を含む座標近傍
        $(U,\varphi)$, $(W, \omega)$の存在を
        示せた. 
        \item[(2)] 
        $\omega \circ (g\circ f)\circ \varphi^{-1}$
        は
        $$\omega \circ (g\circ f)\circ \varphi^{-1}
        =(\omega \circ g\circ \psi^{-1})\circ
        (\psi \circ f\circ \varphi^{-1})$$
        と分解される. 
        $g$は$C^s$級写像より, 
        $\omega \circ g\circ \psi^{-1}$
        は$C^s$級, 
        $f$は$C^s$級写像より,
        $\psi \circ f\circ \varphi^{-1}$
        は$C^s$級であるから, 
        $\omega \circ (g\circ f)\circ \varphi^{-1}$
        は$C^s$級となる. 
    \end{itemize}
    以上より, 合成写像$g\circ f:M\to Q$
    は$C^s$級である.
\end{proof}
\begin{definition}\label{def:C^s deffeomorphism}
    $M$, $N$を$C^r$級多様体とする. $f:M\to N$
    が$C^s$級微分同相写像であるとは, 次の条件
    (1), (2)を満たすことをいう. 
    \begin{itemize}
        \item[(1)]
        $f:M\to N$は全単射($1$対$1$かつ上への写像)
        である. 
        \item[(2)] 
        $f:M\to N$と$f^{-1}:N\to M$はともに$C^s$
        級写像である. 
    \end{itemize}
\end{definition}
\begin{proposition}\label{prop: cord-nabor condition}
    $M$を$m$次元$C^r$級多様体, $V$を$M$の開集合, 
    $V'$を$\mathbb{R}^m$の開集合とする. また, 
    $\varphi:V\to V'$を同相写像とする. 
    このとき, $(V,\varphi)$が$M$の$C^r$級
    座標近傍になるための必要十分条件は, 
    $\varphi:V\to V'$が, $C^r$級微分同相写像
    であることである. 
\end{proposition}
\begin{proof}
    まず, $(V,\varphi)$が$C^r$級座標近傍なら, 
    $\varphi:V\to V'$が, $C^r$級微分同相写像
    であることを証明する. 
    $\varphi ^{-1}\circ\varphi:V'\to V'$
    が$C^r$級であることを示せばよい.  

\end{proof}
%
\section{接ベクトル空間}
\subsection{接ベクトル空間}
$M$を$m$次元$C^r$級多様体とし 
($1\leq r\leq \infty$), $M$の$1$点$p$を通る
$C^r$級曲線$c$を考える. ここで, $c$のパラメータ
$t$は$-\epsilon \leq t\leq \epsilon$の
範囲を動くとし, $c(0)=p$であるとする. 
\begin{definition}\label{def:directional derivative}
    点$p$における方向微分$\boldsymbol{v}$とは, 
    点$p$の開近傍上で
    定義された$C^r$級関数$f$に実数
    $\boldsymbol{v}(f)$を
    対応させる操作であって, 次の性質(0), (1), (2)
    をもつものである. 
    \begin{itemize}
        \item[(0)]$f$と$g$が点$p$の十分
        小さな近傍で上で一致すれば, 
        $\boldsymbol{v}(f)=\boldsymbol{v}(g)$. 
        \item[(1)]$\boldsymbol{v}(af+bg)
        =a\boldsymbol{v}(f)+
        a\boldsymbol{v}(g)$($a,b\in \mathbb{R}$)
        \item[(2)]$\boldsymbol{v}(fg)=
        \boldsymbol{v}(f)g(p)+
        \boldsymbol{v}(g)f(p)$
    \end{itemize}
\end{definition}
    \begin{example}
        $c$に沿う($t=0$における)方向微分
        点$p\in M$の開近傍$U$で定義された任意の
        $C^r$級関数$f$を考える. 
        $c:(-\epsilon, \epsilon)\to U$
        と$f:U\to \mathbb{R}$の合成関数
        $f(c(t))$の$t=0$における
        通常の微分係数
        $$\left .\frac{df(c(t))}{dt}\right|_{t=0}
        \ \left( \lim_{t\to 0}
        \frac{f(c(t))-f(c(0))}{t} \right) $$
        を考える. 
        関数$f$に上の微分係数を対応させる対応
        $$\left .\frac{dc}{dt}\right|_{t=0}:
        f\mapsto 
        \left .\frac{df(c(t))}{dt}\right|_{t=0}$$
        を曲線$c$の$t=0$における速度ベクトルという. 
        曲線$c$の$t=0$における速度ベクトルは
        方向微分である. 実際, 
        \begin{itemize}
            \item[(0)]$f$, $g$が点$p$の開近傍上で定義
            された$C^r$級関数で, $p$のある十分
            小さな開近傍上で$f=g$ならば, 
            $$\left .\frac{dc}{dt}\right|_{t=0}
            (f)=\left .\frac{df(c(t))}{dt}\right|_{t=0}=
            \left .\frac{dg(c(t))}{dt}\right|_{t=0}=
            \left .\frac{dc}{dt}\right|_{t=0}(g)$$
            \item[(1)]$f$, $g$が点$p$の開近傍上で定義
            された$C^r$級関数で, $a, b\in \mathbb{R}$
            のとき, 
            \begin{eqnarray*}
                \left .\frac{dc}{dt}\right|_{t=0}
            (af+bg)&=&
            \left .\frac{d(af(c(t))+bg(c(t)))}{dt}
            \right|_{t=0}\\
            &=&
            a\left .\frac{df(c(t))}{dt}\right|_{t=0}+
            b\left .\frac{dg(c(t))}{dt}\right|_{t=0}\\
            &=&
            a\left .\frac{dc}{dt}\right|_{t=0}(f)+
            b\left .\frac{dc}{dt}\right|_{t=0}(g)
            \end{eqnarray*}
            \item[(2)]
            $f$, $g$が点$p$の開近傍上で定義
            された$C^r$級関数で, $fg$をそれらの
            積とすれば, 
            \begin{eqnarray*}
                \left .\frac{dc}{dt}\right|_{t=0}
            (fg)&=&
            \left\{\frac{df(c(t))}{dt}g(c(t))+
            f(c(t))\frac{dg(c(t))}{dt}
            \right\}_{t=0}\\
            &=&
            \left .\frac{df(c(t))}{dt}\right|_{t=0}
            g(p)+ f(p)
            \left .\frac{dg(c(t))}{dt}\right|_{t=0}\\
            &=&
            \left .\frac{dc}{dt}\right|_{t=0}(f)g(p)
            +f(p)
            \left .\frac{dc}{dt}\right|_{t=0}(g)
            \end{eqnarray*}
        \end{itemize}
    \end{example}
    \begin{example}
        $p$を含む座標近傍$(U;x_1,\cdots x_m)$を
        $1$つ固定する. $p$のまわりで定義された
        $C^r$級関数$f$に, $p$における$x_i$
        方向の偏微分係数を対応させる操作を
        $\left(\frac{\partial}{\partial x_i}
        \right)_p$と書く. すなわち, 
        $$\left(\frac{\partial}{\partial x_i}
        \right)_p:f\mapsto 
        \frac{\partial f}{\partial x_i}(p)$$. 
        このとき, $\left(\frac{\partial}
        {\partial x_i}\right)_p$は$p$における
        方向微分の性質(0), (1), (2)を満たす. 
    \end{example}
    \begin{definition}
        点$p\in M$における方向微分全ての集合を
        $$D_p^r(M)$$
        と表す. $D_p^r(M)$の$2$元$u$, $v$と
        実数$a$に対して, 
        和$\boldsymbol{u}+\boldsymbol{v}$と
        実数倍$a\boldsymbol{u}$を
        $$(\boldsymbol{u}+\boldsymbol{v})(f)=
        \boldsymbol{u}(f)+\boldsymbol{v}(v)$$
        $$a\boldsymbol{u}(f)=
        a(\boldsymbol{u}(f))$$
        と定めると, $D_p^r(M)$は
        $\mathbb{R}$上のベクトル空間となる. 
        ただし, $D_p^r(M)$のゼロベクトルは
        任意の$f$に$0$を対応させる自明な
        方向微分$\boldsymbol{0}$である. 
    \end{definition}
    \begin{proposition}
        ベクトル空間$D_p^r(M)$の元として, 
        $m$個のベクトル
        $\left(\frac{\partial}{\partial x_1}\right)_p, 
        \cdots 
        \left(\frac{\partial}{\partial x_m}\right)_p$
        は$1$次独立である. 
    \end{proposition}
    \begin{proof}
        実数$a_1,\cdots a_m$に対して
        $a_1\left(\frac{\partial}{\partial x_1}\right)_p+
        \cdots +
        a_m\left(\frac{\partial}{\partial x_m}\right)_p=
        \boldsymbol{0}$
        ならば, $a_1=\cdots =a_m=0$であることを
        示せばよい. 
        $a_1\left(\frac{\partial}{\partial x_1}\right)_p+
        \cdots +
        a_m\left(\frac{\partial}{\partial x_m}\right)_p=
        \boldsymbol{0}$
        と仮定すると, 点$p$のまわりで定義された
        任意の$C^r$級関数$f$について
        $a_1\left(\frac{\partial}{\partial x_1}\right)_p(f)+
        \cdots +
        a_m\left(\frac{\partial}{\partial x_m}\right)_p(f)=
        0$
        が成り立つ. つまり, 
        $a_1\frac{\partial f}{\partial x_1}(p)+
        \cdots +
        a_m\frac{\partial f}{\partial x_m}(p)=
        0$
        となる. $f$は任意であったから, $f=x_1$
        という関数をとってくると, 
        \begin{eqnarray*}
            \frac{\partial f}{\partial x_1}&=&1\\
            \frac{\partial f}{\partial x_i}&=&0
            \ \text{($i\geq 2$)}
        \end{eqnarray*}
        であるから, 最初の項のみが$a_1$となって残り, 
        あとは消える. よって$a_1=0$がわかる. 
        同様に$f=x_2,\cdots x_m$を次々に代入
        して, $a_1=\cdots =a_m=0$が示される. 
        これで命題が証明された. 
    \end{proof}
    \begin{definition}\label{def:tangent vector space}
        $m$個のベクトル
        $\left(\frac{\partial}{\partial x_1}\right)_p, 
        \cdots 
        \left(\frac{\partial}{\partial x_m}\right)_p$
        の張る$D_p^r(M)$の部分ベクトル空間を, 点$p$
        における$M$の接ベクトル空間とよび, 
        $$T_p(M)$$
        という記号で表す. 接ベクトル空間$T_p(M)$の
        元を接ベクトルとよぶ. 
    \end{definition}
    \begin{proposition}
        $T_p(M)$は点$p$のまわりの局所座標系のとり方に
        よらず一意に定まる. つまり, 点$p$のまわりで
        $(x_1,\cdots ,x_m)$と別の局所座標系
        $(y_1,\cdots ,y_m)$を選んだとき, 
        方向微分
        $$\left(\frac{\partial}{\partial y_1}\right)_p, 
        \cdots ,
        \left(\frac{\partial}{\partial y_m}\right)_p$$
        の張る$D_p^r(M)$の部分ベクトル空間は, 
        $$\left(\frac{\partial}{\partial x_1}\right)_p, 
        \cdots ,
        \left(\frac{\partial}{\partial x_m}\right)_p$$
        の張る部分ベクトル空間に一致する. 
    \end{proposition}
    \begin{proof}
        点$p$のまわりで定義された任意の$C^r$級
        関数$f$について, 次の変換公式が成り立つ. 
        $$\frac{\partial f}{\partial x_i}
        =\sum_{j=1}^{m}\frac{\partial y_j}
        {\partial x_i}(p)
        \frac{\partial f}{\partial y_j}(p).$$
        方向微分の記号を使ってかくと, 
        $$\left(\frac{\partial}{\partial x_i}\right)_p=
        \sum_{j=1}^{m}
        \frac{\partial y_j}{\partial x_i}(p)
        \left(\frac{\partial}{\partial y_j}\right)_p$$
        を得る. 右辺の
        $\frac{\partial y_j}{\partial x_i}(p)$
        を単なる実数と思うと, この式は, 
        方向微分$\left(\frac{\partial}{\partial x_i}\right)_p$
        ($i=1,\cdots m$)が
        $\left(\frac{\partial}{\partial y_1}\right)_p, \cdots 
        ,\left(\frac{\partial}{\partial y_m}\right)_p$
        の$1$次結合で表されるということを意味している. 
        したがって, 
        $\left(\frac{\partial}{\partial x_1}\right)_p, \cdots 
        ,\left(\frac{\partial}{\partial x_m}\right)_p$
        は
        $\left(\frac{\partial}{\partial y_1}\right)_p, \cdots 
        ,\left(\frac{\partial}{\partial y_m}\right)_p$
        の張るベクトル空間に属している. 
        $(x_1,\cdots ,x_m)$と$(y_1,\cdots ,y_m)$
        の役割を入れ換えて同じ議論をすれば, 逆に
        $\left(\frac{\partial}{\partial y_1}\right)_p, \cdots 
        ,\left(\frac{\partial}{\partial y_m}\right)_p$
        が
        $\left(\frac{\partial}{\partial x_1}\right)_p, \cdots 
        ,\left(\frac{\partial}{\partial x_m}\right)_p$
        の張るベクトル空間に属していることも証明される. 
        したがって, 両者の張るベクトル空間は一致する. 
    \end{proof}
    上の証明中に出てきた基底の変換公式は重要であるから
    改めて命題の形で述べておく. 
    \begin{proposition}\label{prop:basis conversion}
        点$p$のまわりに$2$つの局所座標系
        $(x_1,\cdots x_m)$, $(y_1,\cdots ,y_m)$
        があると, それに応じて$T_p(M)$の基底
        $$\left<\left(\frac{\partial}
        {\partial x_1}\right)_p,\cdots ,
        \left(\frac{\partial}
        {\partial x_m}\right)_p\right>,\ 
        \left<d\left(\frac{\partial}
        {\partial y_1}\right)_p,\cdots ,
        \left(\frac{\partial}
        {\partial y_m}\right)_p\right>$$
        が定まる. それらの間には次の関係式
        が成り立つ. 
        $$\left(\frac{\partial}{\partial x_i}\right)_p=
        \sum_{j=1}^{m}
        \frac{\partial y_j}{\partial x_i}(p)
        \left(\frac{\partial}{\partial y_j}\right)_p$$
    \end{proposition}
    \begin{proposition}\label{prop:dc/dt is tangent vector}
        曲線$c$の$t=0$における速度ベクトル
        $\left .\frac{dc}{dt}\right|_{t=0}$
        は$T_p(M)$の元, すなわち接ベクトルである. 
    \end{proposition}
    \begin{proof}
        点$p$のまわりの局所座標系$(x_1,\cdots ,x_m)$
        を任意に固定する. 
        $\left .\frac{dc}{dt}\right|_{t=0}$
        が$\left(\frac{\partial}{\partial x_1}\right), 
        \cdots ,
        \left(\frac{\partial}{\partial x_m}\right)$
        という形の方向微分の$1$次結合として表されている
        ことを証明すればよい. 
        $p$のまわりで定義された任意の$C^r$級関数$f$を
        をとり, 関数$f$と曲線$c$を局所座標表示したものが
        それぞれ, 
        $$f(x_1,\cdots ,x_m),\ 
        c(t)=(x_1(t),\cdots x_m(t))$$
        であったとすると, 
        \begin{eqnarray*}
            \left .\frac{dc}{dt}\right|_{t=0}(f)&=&
            \left .\frac{d}{dt}f(x_1(t),\cdots ,x_m(t))
            \right|_{t=0}\\
            &=&\sum_{i=1}^{m}\frac{dx_i}{dt}(0)
            \frac{\partial f}{\partial x_i}(p)
        \end{eqnarray*}
        と計算される(合成関数の微分法). 
        $f$は任意であったから, 方向微分としての等式
        $$\left .\frac{dc}{dt}\right|_{t=0}=
        \sum_{i=1}^{m}\frac{dx_i}{dt}(0)
            \left(\frac{\partial}{\partial x_i}
            \right)_p$$
        を得る. 右辺の$\frac{dx_1}{dt}(0), 
        \cdots ,\frac{dx_m}{dt}(0)$
        を単なる実数と思うと, 
        $\left .\frac{dc}{dt}\right|_{t=0}$
        が
        $\left(\frac{\partial}{\partial x_1}\right), 
        \cdots ,
        \left(\frac{\partial}{\partial x_m}\right)$
        の$1$次結合で表されることがわかる. 
        これで命題が証明された. 
    \end{proof}

\subsection{$C^r$級写像の微分}
$M$, $N$をそれぞれ$m$次元, $n$次元の$C^r$級
多様体, $f:M\to N$を
$C^r$級写像とし($1 \leq r\leq \infty$), 
点$p\in M$を通る$M$上の$C^r$曲線を
$$c:(-\epsilon, \epsilon)\to M,\ c(0)=p$$
とする. この曲線を写像$f$でうつすと, 点
$q=f(p)$を通る$N$上の$C^r$曲線
$$f\circ c:(-\epsilon, \epsilon)\to N, f^\circ c(0)=q$$
が得られる. 
$t=0$における曲線$c$の速度ベクトル
$$\left .\frac{dc}{dt}\right|_{t=0}\in T_p(M)$$
と, $t=0$における曲線$f\circ c$の速度ベクトル
$$\left .\frac{d(f\circ c)}{dt}\right|_{t=0}
\in T_q(M)$$
の関係を調べる. 
\begin{proposition}\label{prop:relation of v and w}
    $t=0$における$c$と$f\circ c$の速度ベクトルを
    それぞれ
    $$\left .\frac{dc}{dt}\right|_{t=0}=
    \sum_{i=1}^{m}v_i\left(\frac{\partial}
    {\partial x_i}\right)_p,\ 
    \left .\frac{d(f\circ c)}{dt}\right|_{t=0}=
    \sum_{j=1}^{n}w_j\left(\frac{\partial}
    {\partial y_j}\right)_q$$
    とおくと, 係数$v_1,\cdots ,v_m$と
    $w_1,\cdots ,w_n$の間には次の関係がある. 
    $$w_j=\sum_{i=1}^{m}\frac{\partial f_j}
    {\partial x_i}(p)v_i\ (j=1,\cdots ,n)$$
    またこの式を, 縦ベクトルと行列を使って書くと
    $$\begin{pmatrix}
          w_1 \\
          \vdots \\
          w_n 
        \end{pmatrix}
        =
        \left(
        \begin{array}{ccc}
          \frac{\partial f_1}{\partial x_1}(p)&\cdots &\frac{\partial f_1}{\partial x_m}(p)\\
          \vdots &\ddots& \vdots \\
          \frac{\partial f_n}{\partial x_1}(p)&\cdots &\frac{\partial f_n}{\partial x_m}(p) 
        \end{array} 
        \right)=
        \begin{pmatrix}
          v_1\\
          \vdots \\
          v_m
        \end{pmatrix}
        $$
    となる. 
\end{proposition}
\begin{proof}
    点$p$を含む$M$の座標近傍$(U;x_1,\cdots x_m)$
    と, $q=f(p)$を含む$N$の座標近傍
    $(V;y_1,\cdots y_n)$を$f(U)\subset V$
    となるようにとる. $f$を$(U;x_1,\cdots x_m)$
    と$(V;y_1,\cdots y_n)$に関して局所座標
    表示したものが
    \begin{eqnarray*}
        y_1&=&f_1(x_1,\cdots ,x_m)\\
        &\vdots& \\
        y_n&=&f_n(x_1,\cdots ,x_m)
    \end{eqnarray*}
    であるとする. 右辺の関数は$(x_1,\cdots x_m)$
    に関して$C^r$級である. 
    パラメータ$t$の動く範囲$-\epsilon\leq t \epsilon$
    を十分小さくとれば, 曲線$c$は
    $(U;x_1,\cdots x_m)$の中に含まれるとしてよい. 
    局所座標系$(x_1,\cdots ,x_m)$に関する$c$の
    局所座標表示を
    $$c(t)=(x_1(t),\cdots ,x_m(t))$$
    とすると, 曲線$f\circ c$の$(y_1,\cdots ,y_n)$
    に関する局所座標表示は
    \begin{eqnarray*}
        f\circ c(t)&=&(y_1(t),\cdots ,y_n(t))\\
        &=&(f_1(x_1(t),\cdots ,x_m(t)),\cdots ,
        f_n(x_1(t),\cdots ,x_m(t)))
    \end{eqnarray*}
    で与えられる. 
    曲線$c$の$t=0$における速度ベクトルを
    $$\left .\frac{dc}{dt}\right|_{t=0}=
    v_1\left(\frac{\partial}{\partial x_1}\right)_p
    +\cdots +
    v_m\left(\frac{\partial}{\partial x_m}\right)_p$$
    とおく. ただし$v_1,\cdots ,v_m$は実数である. 
    命題\ref{prop:dc/dt is tangent vector}より, 
    $v_i=\frac{dx_i}{dt}(0)\ (i=1,\cdots ,m)$
    である. 曲線$f\circ c$の速度ベクトルを求めると
    \begin{eqnarray*}
        \left .\frac{f\circ c}{dt}\right|_{t=0}
        &=&\sum_{j=1}^{n}\frac{dy_j}{dt}(0)
        \left(\frac{\partial}{\partial y_j}
        \right)_q\\
        &=&
        \sum_{j=1}^{n}\left .
        f_j(x_1(t),\cdots ,x_m(t))\right|_{t=0}
        \left(\frac{\partial}{\partial y_j}
        \right)_q\\
        &=&
        \sum_{j=1}^{n}\left\{
            \sum_{i=1}^{m}\frac{\partial f_j}
            {\partial x_i}(p)
            \frac{dx_i}{dt}(0)
        \right\}
        \left(\frac{\partial}{\partial y_j}
        \right)_q\\
        &=&
        \sum_{j=1}^{n}\left\{
            \sum_{i=1}^{m}\frac{\partial f_j}
            {\partial x_i}(p)
            v_i
        \right\}
        \left(\frac{\partial}{\partial y_j}
        \right)_q
    \end{eqnarray*}
    となる. 速度ベクトル
    $\left .\frac{f\circ c}{dt}\right|_{t=0}$
    の係数を$w_1, \cdots ,w_n$とすると, 
    $$w_j=\sum_{i=1}^{m}\frac{\partial f_j}
    {\partial x_i}(p)v_i\ (j=1,\cdots ,n)$$
    を得る. よって命題は証明された. 
\end{proof}
\begin{definition}\label{def:Jacobian matrix}
    命題\ref{prop:relation of v and w}
に現れた$n$行$m$列の行列
$$\left(
    \begin{array}{ccc}
      \frac{\partial f_1}{\partial x_1}(p)&\cdots &\frac{\partial f_1}{\partial x_m}(p)\\
      \vdots &\ddots& \vdots \\
      \frac{\partial f_n}{\partial x_1}(p)&\cdots &\frac{\partial f_n}{\partial x_m}(p) 
    \end{array} 
  \right)$$
を, 点$p$における写像$f:M\to N$のヤコビ行列
とよび, 記号で
$$(Jf)_p$$
と表す. ヤコビ行列$(Jf)_p$は, 局所座標系
$(x_1,\cdots ,x_m)$と$(y_1,\cdots ,y_n)$
を選ぶことによって決まる行列である. 
\end{definition}
命題\ref{prop:relation of v and w}より次の
系が分かる. 
\begin{corollary}\label{coro:any c is OK}
    $\left .\frac{d(f\circ c)}{dt}\right|_{t=0}$
    は$\left .\frac{dc}{dt}\right|_{t=0}$
    のみに依存して定まり, 点$p$から離れたところ
    での$c$の振る舞いによらない. 
    すなわち, 点$p\in M$を通る曲線$2$本を
    $$c:(-\epsilon, \epsilon)\to M\ (c(0)=p)$$
    $$c':(-\epsilon, \epsilon)\to M\ (c'(0)=p)$$
    とし, 
    $\left .\frac{dc}{dt}\right|_{t=0}=
    \left .\frac{dc'}{dt}\right|_{t=0}$
    と仮定すると, 
    $\left .\frac{d(f\circ c)}{dt}\right|_{t=0}=
    \left .\frac{d(f\circ c')}{dt}\right|_{t=0}$
    が成り立つ. なぜなら$f$を固定しておけば, 
    ベクトル$(w_1,\cdots ,w_n)$は, ベクトル
    $(v_1,\cdots ,v_m)$のみから計算
    されるからである. 
\end{corollary}
\begin{proposition}\label{prop:c exist}
    $T_p(M)_p$に属する任意のベクトル$\boldsymbol{v}$
    に対し, 点$p$を通る$C^r$級曲線
    $$c:(-\epsilon, \epsilon)\to M\ (c(0)=p)$$
    が存在して, $\left .\frac{dc}{dt}
    \right|_{t=0}=\boldsymbol{v}$
    が成り立つ. 
\end{proposition}
\begin{proof}
    $p$のまわりの局所座標系$(x_1,\cdots ,x_m)$
    を固定して考える. このとき, 
    $$\boldsymbol{v}=
    v_1\left(\frac{\partial}{\partial x_1}\right)_p
    +\cdots +
    v_m\left(\frac{\partial}{\partial x_m}\right)_p$$
    であるとする. これに応じて, 曲線$c$を
    $$c(t)=(a_1+v_1t,\cdots ,a_m+v_mt)$$
    と定義すればよい. ここに, $(a_1,\cdots ,a_m)$
    は$P$の局所座標である. 速度ベクトル
    $\left .\frac{dc}{dt}\right|_{t=0}$
    を計算すれば, $\boldsymbol{v}$に一致する
    ことが確かめられる. 
\end{proof}
系\ref{coro:any c is OK}と命題\ref{prop:c exist}
により, 次のような写像
$$T_p(M)\to T_q(N),\ 
\left .\frac{dc}{dt}\right|_{t=0}\mapsto
\left .\frac{d(f\circ c)}{dt}\right|_{t=0}
\ (q=f(p))$$
が自然に定義できる. 
\begin{definition}\label{def:differential}
    こうして得られた写像を
    $$(df)_p:T_p(M)\to T_q(N)$$
    と書き, 点$p$における$f:M\to N$
    の微分とよぶ. 
\end{definition}
\begin{proposition}
    $(df)_p:T_p(M)\to T_p(M)$は線型写像である. 
\end{proposition}
\begin{proof}
    線型写像とはベクトルの和をベクトルの和にうつし, 
    ベクトルの$a$倍をベクトルの$a$倍にうつすような写像
    である. 
    $p$, $q$のまわりにそれぞれ局所座標系
    $(x_1,\cdots ,x_m)$, $(y_1,\cdots ,y_n)$
    をとり, $T_p(M)$, $T_q(N)$の中に
    基底$\left< \left(\frac{\partial}
    {\partial x_1}\right)_p,\cdots ,
    \left(\frac{\partial}
    {\partial x_m}\right)_p\right>$, 
    $\left< \left(\frac{\partial}
    {\partial y_1}\right)_p,\cdots ,
    \left(\frac{\partial}
    {\partial y_n}\right)_p\right>$
    を固定すると, 
    任意の$\boldsymbol{v}\in T_p(M)$は
    $$\boldsymbol{v}=
    \sum_{i=1}^{m}v_i
    \left(\frac{\partial}{\partial x_i}
    \right)_p$$
    と書け, 数ベクトル$(v_1,\cdots ,v_m)$
    と$1$対$1$に対応する. 同様に, 
    任意の$\boldsymbol{w}\in T_q(N)$は
    $$\boldsymbol{w}=
    \sum_{j=1}^{n}v_i
    \left(\frac{\partial}{\partial y_j}
    \right)_q$$
    と書け, 数ベクトル$(w_1,\cdots ,w_n)$
    と$1$対$1$に対応する. 今, 
    $\boldsymbol{w}=(df)_p(\boldsymbol{v})$
    であるとすると, 
    $\boldsymbol{w}$と$\boldsymbol{v}$
    に対応する数ベクトルを縦ベクトルに書いたとして, 
    $$\begin{pmatrix}
        w_1\\
        \vdots \\
        w_n
    \end{pmatrix}=(Jf)_p
    \begin{pmatrix}
        v_1\\
        \vdots \\
        v_m
    \end{pmatrix}$$
    の関係がある. $(Jf)_p$は点$p$における$f$
    のヤコビ行列である. この関係から
    $(df)_p:\boldsymbol{v}\mapsto 
    \boldsymbol{w}$が線型写像であることがわかる. 
\end{proof}
\begin{proposition}
    点$p$, $q$のまわりでそれぞれ局所座標系
    $(x_1,\cdots ,x_m)$, $(y_1,\cdots ,y_n)$
    を固定し, それによって$f$を
    \begin{eqnarray*}
        y_1&=&f_1(x_1,\cdots ,x_m)\\
        &\vdots&\\
        y_n&=&f_n(x_1,\cdots ,x_m)
    \end{eqnarray*}
    と局所座標表示する. このとき, 次の
    公式が成り立つ. 
    $$(df)_p\left( \left(
        \frac{\partial}{\partial x_i}
    \right)_p\right)=
    \sum_{j=1}^{n}\frac{\partial f_j}
    {\partial x_i}(p)\left(
        \frac{\partial}{\partial y_j}
    \right)_q$$
\end{proposition}
\begin{proof}
    $\left(\frac{\partial}{\partial x_i}\right)_p$
    を$\sum_{i=1}^{m}v_i\left(\frac
    {\partial}{\partial x_i}\right)_p$
    の形で表すと, $v_i=1,\ v_k=0\ (k\neq i)$
    である. 
    $(df)_p\left( \left(
        \frac{\partial}{\partial x_i}
    \right)_p\right)=
    \sum_{j=1}^{n}w_j\left(
        \frac{\partial}{\partial y_j}
    \right)_q$
    とおいたときの$w_j$は
    $$w_j=\sum_{i=1}^{m}v_i \frac{\partial f_j}
    {\partial x_i}(p)$$
    であるから, $v_i=1,\ v_k=0\ (k\neq i)$
    を代入して, 
    $$w_j=v_i \frac{\partial f_j}
    {\partial x_i}(p)$$
    を得る. よって, 
    $$(df)_p\left( \left(
        \frac{\partial}{\partial x_i}
    \right)_p\right)=
    \sum_{j=1}^{n}w_j\left(
        \frac{\partial}{\partial y_j}
    \right)_q=
    \sum_{j=1}^{n}\frac{\partial f_j}
    {\partial x_i}(p)\left(
        \frac{\partial}{\partial y_j}
    \right)_q$$
    が成り立つことが分かる. 
\end{proof}
\begin{proposition}
    $f:M\to N$を$C^r$級写像とする. $M$の点$p$
    における任意の接ベクトル$\boldsymbol{v}
    \in T_p(M)$と, $N$の点$f(p)$のまわりで定義された
    任意の$C^r$級関数$\xi$について
    $$((df)_p(\boldsymbol{v}))(\xi)=
    \boldsymbol{v}(\xi \circ f)$$
    が成り立つ. 
\end{proposition}
\begin{proof}
    $\boldsymbol{v}=\left .\frac{dc}{dt}
    \right|_{t=0}$であるような$C^r$級
    曲線$c:(-\epsilon, \epsilon)\to M$, 
    $c(0)=p$をとる. $(df)_p$の定義より, 
    $$(df)_p(\boldsymbol{v})=
    \left .\frac{d(f\circ c)}{dt}\right|_{t=0}$$
    である. 
    速度ベクトル$\left .\frac{d(f\circ c)}{dt}\right|_{t=0}$
    を方向微分と考えると, 
    $$\left .\frac{d(f\circ c)}{dt}\right|_{t=0}
    (\xi)=
    \left .\frac{d\xi(f\circ c(t))}
    {dt}\right|_{t=0}$$
    が成り立つ. よって, 
    \begin{eqnarray*}
        (df)_p(\boldsymbol{v})(\xi)&=&
        \left .\frac{d\xi(f\circ c(t))}
        {dt}\right|_{t=0}(\xi)\\
        &=&\left .\frac{d\xi(f\circ c(t))}
        {dt}\right|_{t=0}\\
        &=&\left .\frac{d(\xi\circ f(c(t)))}
        {dt}\right|_{t=0}\\
        &=&
        \left .\frac{dc}{dt}\right|_{t=0}(\xi 
        \circ f)\\
        &=&\boldsymbol{v}(\xi\circ f)
    \end{eqnarray*}
    となり, 命題は証明された. 
\end{proof}
\begin{proposition}
    $M$, $N$, $Q$をそれぞれ, $m$次元, $n$次元, 
    $q$次元の$C^r$級多様体, 
    $f:M\to N$, $g:N\to Q$を$C^r$級写像, 
    $p$を$M$の点とする($1\leq r\leq \infty$). 
    このとき
    $$d(g\circ f)_p=(dg)_{f(p)}\circ (df)_p:
    T_p(M)\to T_{g\circ f(p)}(Q)$$
    が成り立つ. 
\end{proposition}
\begin{proof}
    任意の接ベクトル$\boldsymbol{v}\in T_p(M)$
    をとる. $\boldsymbol{v}=
    \left .\frac{dc}{dt}\right|_{t=0}$であるような
    $C^r$級曲線$c:(-\epsilon, \epsilon)\to M$, 
    $c(0)=p$をとる. 
    \begin{eqnarray*}
        d(g\circ f)_p(\boldsymbol{v})&=&
        d(g\circ f)_p\left(\left .
        \frac{dc}{dt}\right|_{t=0}\right)\\
        &=&\left .
        \frac{d(g\circ f)\circ c}{dt}
        \right|_{t=0}\\
        &=&\left .
        \frac{dg\circ (f\circ c)}{dt}
        \right|_{t=0}\\
        &=&(dg)_{f(p)}
        \left(\left .
        \frac{d(f\circ c)}{dt}\right|_{t=0}\right)\\
        &=&(dg)_{f(p)}\circ (df)_p
        \left(\left .
        \frac{dc}{dt}\right|_{t=0}\right)\\
        &=&(dg)_{f(p)}\circ (df)_p
        (\boldsymbol{v})
    \end{eqnarray*}
    となる. ここで, $\boldsymbol{v}\in T_p(M)$は任意
    であったから, $d(g\circ f)_p=
    (dg)_{f(p)}\circ(df)_p$が成り立つ. 
\end{proof}
$(df)_p$, $(dg)_{f(p)}$, $d(g\circ f)_p$はそれぞれ
ヤコビ行列$(Jf)_p$, $(Jg)_{f(p)}$, $J(g\circ f)_p$
で表現されるから, 次の系が得られる. 
\begin{corollary}
    $J(g\circ f)_p=(Jg)_{f(p)}(Jf)_p$である. 
\end{corollary}
\begin{corollary}
    \begin{itemize}
        \item[(i)]$M=N$のとき, $d(id_M)_p
        =id_{T_p(M)}$である($d(id_M)_p$, 
        $id_{T_p(M)}$はそれぞれ$M$, $T_p(M)$の
        恒等写像). 
        \item[(ii)]
        $f:M\to N$が$C^r$級微分同相写像なら, 任意の
        $p\in M$について, $(df)_p:T_p(M)\to T_{f(p)}(N)$
        は線型写像として同型であって, 
        $$(df^{-1})_{f(p)}=(df)_p^{-1}$$
        が成り立つ. 
    \end{itemize}
\end{corollary}
この系をヤコビ行列を使って書き直すと
次の系が得られる. 
\begin{corollary}
    \begin{itemize}
        \item[(i)]$M=N$のとき, $p$のまわりの
        局所座標系$(x_1,\cdots ,x_m)$を固定
        すれば, $J(id_M)_p=E_m$である
        ($E_m$は$m$次単位行列). 
        \item[(ii)]
        $f:M\to N$が$C^r$級微分同相写像なら, 
        $p$, $f(p)$のまわりで局所座標系
        $(x_1,\cdots x_m)$, $(y_1,\cdots ,y_n)$
        を固定すれば, $(Jf)_p$は正則行列, 
        すなわち, 逆行列をもつ正方行列であって, 
        $$(Jf^{-1})_{f(p)}=(Jf)_p^{-1}$$
        が成り立つ. 
    \end{itemize}
\end{corollary}
とくに, 次が分かる. 
\begin{corollary}\label{coro:dim equality by diffeomorphism}
    $C^r$級微分同相写像$f:M\to N$が存在すれば, 
    $M$の次元$m$と$N$の次元$n$は等しい. つまり, 
    $$\text{dim}M=\text{dim}N$$
    が成り立つ. 
\end{corollary}
\newpage


\section{多様体の次元を調べる方法}
\subsection{写像の局所的性質}
\begin{theorem}\label{theo:f^(-1)theorem}
    $(df)_p:T_p(M)\to T_p(N)$が線型写像として
    同型なら, $f$は$p$のある開近傍から$f(p)$の
    ある開近傍への$C^r$級微分同相写像である. 
    すなわち, $p$の開近傍$U$と$f(p)$の開近傍$V$
    が存在して, $f(U)=V$となり, かつ, 
    $f|U:U \to V$は$C^r$級微分同相写像である. 
\end{theorem}
\begin{theorem}\label{theo: projection theorem}
    $f:M\to N$を$C^r$級写像とする. ある点$p\in M$
    における微分$(df)_p:T_p(M)\to T_p(N)$が上への
    線型写像なら, 点$p$付近での$f$の様子は, 射影:
    $\mathbb{R}^{m-n} \times \mathbb{R}^n \to \mathbb{R}^n$, 
    $(x_1, \cdots ,x_m)\mapsto (x_(m-n+1), \cdots ,x_m)$
    と同じである. すなわち, $p$のまわりの局所座標系
    $(x_1,\cdots ,x_m)$と$f(p)$のまわりの局所座標系
    $(y_1, \cdots ,y_n)$をうまく選んで, $f$
    の局所座標表示
    $(y_1, \cdots ,y_n)=(c)$が
    \begin{eqnarray*}
        y_1&=&f_1(x_1,\cdots ,x_m)=x_{m-n+1}\\
        &\vdots& \\
        y_n&=&f_n(x_1,\cdots ,x_m)=x_m
    \end{eqnarray*}
    であるようにできる. 
\end{theorem}
\begin{proof}
    $C^r$級写像$f:M\to N$が与えられており, 
    ある点$p\in M$における微分
    $(df)_p:T_p(M)\to T_p(N)$が上への線型写像
    であるとする. このとき, rank$(df)_p=n$
    (ただし, $m=$dim$M\geq n=$dim$N$. )
    点$p$, 点$f(p)$のまわりに, それぞれ座標近傍
    $(U;x_1,\cdots ,x_m)$, $(V;y_1,\cdots ,y_n)$
    をとり, $f$を局所座標表示すると
    $$(y_1, \cdots ,y_n)=(f_1(x_1,\cdots ,x_m), 
    \cdots ,f_n(x_1,\cdots ,x_m))$$
    ヤコビ行列$(Jf)_p$は$(df)_p$を表す行列で, 
    $n$行$m$列である. rank$(df)_p=n$であるから, 
    $(Jf)_p$から$n$本の列ベクトルを選んで, 作った
    正方行列は正則である. 必要なら適当に列を入れ替えて
    $$B=
    \begin{pmatrix}
        \frac{\partial f_1}{\partial x_{m-n+1}}(p) & \cdots & \frac{\partial f_1}{\partial x_m}(p)  \\
        &\cdots& \\
        \frac{\partial f_n}{\partial x_{m-n+1}}(p) & \cdots & \frac{\partial f_n}{\partial x_m}(p)  \\
    \end{pmatrix}
    $$
    とおくと, det$B \neq 0$であって, 
    $$(Jf)_p=
    \left(
        \begin{array}{ccc|c} 
          * & \cdots & *  &  \\
           &  &   & B \\
          * & \cdots & *  & 
        \end{array} 
        \right)
$$
と仮定してよい. $p$のまわりの座標近傍
$(U;x_1,\cdots ,x_m)$から, $\mathbb{R}^m$への
写像$\varphi :U\to \mathbb{R}^m$を
$$\varphi(x_1,\cdots ,x_m)=(x_1, \cdots ,x_{m-n},f_1(x_1,\cdots ,x_m), 
\cdots ,f_n(x_1,\cdots ,x_m))$$
と定義する. やコビ行列$(J\varphi)_p$は
$$(J\varphi)_p=
    \left(
        \begin{array}{ccc|ccc} 
          1 &        &   & & & \\
            & \ddots &   & &O& \\
            &        & 1 & & &  \\ \hline
          * & \cdots & * & & &  \\
            & \cdots &   & &B& \\
          * & \cdots & * & & &
        \end{array} 
        \right)
$$
となり, det$(J\varphi)_p=$det$B\neq 0$である. 
したがって, 逆関数の定理(定理
\ref{theo:f^(-1) theorem})
により, 点$p$を含む$U$を十分小さくとれば, 
$\varphi |U:U\to \varphi(U)$は$C^r$級微分同相写像
である. よって, /ref{prop: cord-nabor condition}
より,$(U, \varphi|U)$は$p$のまわりの新しい$C^r$
級座標近傍と思える. $(U, \varphi|U)$の
局所座標系を$(z_1,\cdots ,z_m)$とすると, 
上の$\varphi$の定義式から, 
$$(z_1,\cdots ,z_m)=\varphi(x_1,\cdots x_m)
=(x_1,\cdots x_{m-n},f_1(x_1,\cdots ,x_m), 
\cdots ,f_n(x_1,\cdots ,x_m))$$
を得る. とくに, $(z_{m-n+1},\cdots ,z_m)
=(f_1(x_1,\cdots ,x_m), 
\cdots ,f_n(x_1,\cdots ,x_m))$
であるから, 
\begin{eqnarray*}
    f\circ \varphi (z_1,\cdots z_m)&=&f(x_1,\cdots x_m)\\
    &=&(f_1(x_1,\cdots ,x_m),\cdots ,f_n(x_1,\cdots ,x_m))\\
    &=&(z_{m-n+1},\cdots ,z_m)
\end{eqnarray*}
となる. よって, $(z_1,\cdots ,z_m)$と$(y_1,\cdots ,y_n)$
に関する$f:M\to N$の点$p$のまわりでの局所座標表示
$$(z_1,\cdots ,z_m)\mapsto (z_{m-n+1},\cdots ,z_m)
=(y_1,\cdots ,y_n)$$
である. $(z_1,\cdots ,z_m)$を改めて$(x_1,\cdots ,x_m)$
と書き直せば, 定理\ref{theo: projection theorem}の
主張が得られる. 
\end{proof}
\begin{theorem}\label{theo:inclusion map theorem}
    $f:M\to N$を$C^r$級写像とする. 
    $(df)_p:T_p(M)\to T_{f(p)}(N)$が
    $1$対$1$の線型写像なら, 点$p$の付近での$f$の
    様子は, 包含写像
    $\mathbb{R}^m\to \mathbb{R}^n,\ 
    (x_1,\cdots ,x_m)\mapsto 
    (x_1,\cdots ,x_m,0,\cdots ,0)$
    と同じである. すなわち, $p$のまわりの局所座標系
    $(x_1,\cdots ,x_m)$と$f(p)$のまわりの局所座標系
    $(y_1,\cdots ,y_n)$をうまく選んで, $f$の
    局所座標表示
    $(y_1,\cdots ,y_n)\mapsto (f_1(x_1,\cdots x_m),
    \cdots ,f_n(x_1,\cdots x_m),)$が, 
    \begin{eqnarray*}
        y_1&=&f_1(x_1,\cdots x_m)=x_1\\
        &\vdots& \\
        y_m&=&f_m(x_1,\cdots x_m)=x_m\\
        y_{m+1}&=&f_{m+1}(x_1,\cdots x_m)=0\\
        &\vdots& \\
        y_n&=&f_n(x_1,\cdots x_m)=0\\
    \end{eqnarray*}
    であるようにできる. 
\end{theorem}
\begin{proof}
    問題は局所的であるから$M=\mathbb{R}^m$, 
    $N=\mathbb{R}^n$と仮定してよい. ($m\leq n$)
    また, $p=\boldsymbol{o}$, $f(p)=\boldsymbol{o}$
    としてよい. 
    $\mathbb{R}^m$の自然な座標$(x_1, \cdots ,x_m)$
    と, $\mathbb{R}^n$の自然な座標に関して, 
    与えられた$C^r$級写像$f:\mathbb{R}^m\to \mathbb{R}^n$
    を局所座標表示したものを
    $$u_1=f_1(x_1,\cdots ,x_m),\ \cdots ,
    u_n=f_n(x_1,\cdots ,x_m)$$
    とする. $(df)_{\boldsymbol{o}}:
    T_{\boldsymbol{o}}(\mathbb{R}^m)\to 
    T_{\boldsymbol{o}}(\mathbb{R}^n)$
    は$1$対$1$であるから, ヤコビ行列$(Jf)_{\boldsymbol{o}}$
    から, 適当な$m$行を選び出して作った$m$次正方行列
    は正則である. 必要なら, 
    $(u_1, \cdots ,u_m)$の並び方を変えて, 
    $(Jf)_{\boldsymbol{o}}=
    \begin{pmatrix}
        A \\ \hline
        B
     \end{pmatrix}
     $
     ($A$は$m$行$m$列の正則行列)と仮定してよい. 
     $(x_1, \cdots ,x_m)$に新しく, 
     $(x_{m+1}, \cdots ,x_n)$を付け加えて, 
     $\mathbb{R}^m \times \mathbb{R}^{n-m}$を
     構成し, 新しい写像
     $F:\mathbb{R}^m \times \mathbb{R}^{n-m}
     \to \mathbb{R}^n$
    を, 
    \begin{eqnarray*}
        &&F(x_1,\cdots ,x_m,x_{m+1},\cdots x_n)\\
    &&=(f_1(x_1,\cdots ,x_m),\cdots ,f_m(x_1,\cdots ,x_m), 
    f_{m+1}(x_1,\cdots ,x_m)+x_{m+1}, 
    \cdots ,f_{n}(x_1,\cdots ,x_m)+x_{n})
    \end{eqnarray*}
    と定義する.$F$のヤコビ行列
    $$(Jf)_{\boldsymbol{o}}=
    \left(
\begin{array}{c|ccc} 
  A &  & O  &  \\ \hline
   & 1 &   &  \\
  B &  & \vdots &  \\
   &  &   & 1
\end{array} 
\right)
    $$
    であり, 正則である. 
    逆関数の定理\ref{theo:f^(-1)theorem}により, 
    $\mathbb{R}^m \times \mathbb{R}^{n-m}$
    における, $\boldsymbol{o}$の近傍$U$と, 
    $\mathbb{R}^n$における$\boldsymbol{o}$
    の近傍$V$が存在して, $F|U:U\to V$は$C^r$級
    微分同相写像になる. $\psi =(F|U)^{-1}:V\to U$
    とおく. $(V, \psi)$は$\boldsymbol{o}$の
    まわりの$\mathbb{R}^n$の$C^r$級座標近傍と思える. 
    この座標近傍に関する局所座標系を$(y_1,\cdots y_n)$
    とし, $(x_1, \cdots ,x_m)$と$(y_1,\cdots y_n)$に
    関して, $f:U\cap(\mathbb{R}^m\times 
    \{ \boldsymbol{o} \})\to \mathbb{R}^n$
    を局所座標表示すると, 
    \begin{eqnarray*}
        (y_1,\cdots ,y_n)&=&\psi (f(x_1,\cdots ,x_m))\\
        &=&\psi(F(x_1,\cdots ,x_m,0,\cdots 0))\\
        &=&(F|U)^{-1}(F(x_1,\cdots ,x_m,0,\cdots 0))\\
        &=&(x_1,\cdots ,x_m,0,\cdots ,0)
    \end{eqnarray*}
    となる. したがって, 定理\ref{theo:inclusion map theorem}
    の主張する通りの局所座標表示が得られた. 
\end{proof}
\subsection{$C^r$級部分多様体}
\begin{definition}\label{def:C^r-submanifold}
    $n$次元$C^r$級多様体$N$の部分集合$L$が
    $N$の$l$次元$C^r$級部分多様体であるとは, 
    \begin{itemize}
        \item[(1)]$l=n$のとき:$L$が$N$の開集合
        であることである. 
        \item[(2)] $0\leq l<n$のとき:$L$の任意の点$p$
        に対し, $p$を含む$N$の座標近傍$(U;x_1,\cdots ,x_n)$
        が存在して, 
        $$L\cap N=\{(x_1,\cdots ,x_n)\in U|
        x_{l+1}=\cdots =x_n=0\}$$
        が成り立つことである. 
    \end{itemize}
\end{definition}
\begin{proposition}\label{prop:dim of C^r-submanifold}
    $n$次元$C^r$級多様体$N$の$l$次元$C^r$級
    部分多様体$L$は, それ自身$l$次元$C^r$級
    多様体である. 
\end{proposition}
\begin{proof}
    \begin{itemize}
        \item[(1)]
        $l=n$のとき, $N$からの相対位相によって, 
        $L$は位相空間となり, 
        $N$の$C^r$級座標近傍を
        $\mathcal{S}=
        \{U_\alpha,\varphi _\alpha\}_{\alpha\in A}$
        とすると, $\mathcal{S}$を$L$に制限した
        $\{U_\alpha \cap L,
        \varphi _\alpha |U_\alpha \cap L\}_{\alpha\in A}$
        は$L$の$C^r$局所座標系になる. 
        よって, $L$はそれ自身, $l$($=n$)次元
        $C^r$級多様体である. 
        \item[(2)] 
        $0\leq l<n$のとき, $L$には$N$からの相対位相
        を入れる. $N$がハウスドルフ空間であるから$L$
        もそうである. \\
        $L$の任意の点$p$に対し, $p$を含む$N$の
        局所座標系$(U;x_1,\cdots ,x_n)$で定義
        \ref{def:C^r-submanifold}の条件を満たすもの
        を選び, $(U_p;x^p_1,\cdots ,x^p_n)$とする. 
        $V_p=L\cap U_p$とおくと, $V_p$は$L$の開集合
        である. $V_p$上の$U_p$の局所座標系
        $(x^p_1, \cdots ,x^p_n)$の$x^p_1$から
        $x^p_l$までを制限したもの$(V_p;x^p_1,\cdots ,x^p_l)$
        を考える. \\
        $\{(V_p;x^p_1,\cdots ,x^p_l)\}_{p\in L}$が
        $L$を被覆することは明らかである.\\
        $V_p$と$V_q$が交わるとする. 対応する
        $(U_p;x^p_1,\cdots ,x^p_n)$ と
        $(U_q;x^q_1,\cdots ,x^q_n)$は, 
        $N$の適当な座標近傍
        $(U_\alpha ;x^\alpha_1, \cdots ,x^\alpha_n)$,
        $(U_\beta ;x^\beta_1, \cdots ,x^\beta_n)$
        である. この間の座標変換はある$C^r$級関数$f$を
        用いて, 
        $$(x^\beta_1, \cdots ,x^\beta_n)
        =(f_1(x^\alpha_1, \cdots ,x^\alpha_n), \cdots 
        , f_n(x^\alpha_1, \cdots ,x^\alpha_n))$$
        と書ける. $V_p\cap V_q$上では, 
        $x^\alpha_{l+1}=x^\alpha_n=0$, $x^\beta_{l+1}=x^\beta_n=0$
        が成り立つので, $V_p\cap V_q$上では
        \begin{eqnarray*}
            x^\beta_1&=&f_1(x^\alpha_1, \cdots ,x^\alpha_l,0,\cdots ,0)\\
            &\vdots& \\
            x^\beta_l&=&f_l(x^\alpha_1, \cdots ,x^\alpha_l,0,\cdots ,0)\\
            0&=&f_{l+1}(x^\alpha_1, \cdots ,x^\alpha_l,0,\cdots ,0)\\
            &\vdots& \\
            0&=&f_n(x^\alpha_1, \cdots ,x^\alpha_l,0,\cdots ,0)\\
        \end{eqnarray*}
        となっている. 改めて関数$g$を
        \begin{eqnarray*}
            g(x^\alpha_1, \cdots ,x^\alpha_l)&=&
            (g_1(x^\alpha_1, \cdots ,x^\alpha_l), \cdots 
            , g_l(x^\alpha_1, \cdots ,x^\alpha_l))\\
            &=&(f_1(x^\alpha_1, \cdots ,x^\alpha_l,0,\cdots ,0), \cdots 
            , f_l(x^\alpha_1, \cdots ,x^\alpha_l,0,\cdots ,0))
        \end{eqnarray*}
        と定義すると, この関数は$C^r$級である. 
        そして, 
        $$(x^\beta_1, \cdots ,x^\beta_l)
        =g(x^\alpha_1, \cdots ,x^\alpha_l)$$
        が$(V_p;x^p_1,\cdots x^p_l)$から
        $(V_q;x^q_1,\cdots x^q_l)$の座標変換を与えている. \\
        ゆえに, $\{(V_p;x^p_1,\cdots ,x^p_l)\}_{p\in L}$
        は$L$の$C^r$級座標近傍になっている. \\
        以上より, $L$は$l$次元$C^r$級多様体である. 
    \end{itemize}
\end{proof}
\begin{theorem}\label{theo:f^{-1}(q) C^r manifold}
    $M$, $N$を$m$次元, $n$次元の$C^r$級多様体, 
    $f:M\to N$を$C^r$級写像とする. $N$のある点
    $q$について, $f(p)=q$となる$M$の各点$p$
    が常にrank$(Jf)_p=n$を満たすとき, 逆像
    $f^{-1}(q)$は$(m-n)$次元$C^r$級多様体
    である. 
\end{theorem}
\begin{proof}
    定義\ref{def:C^r-submanifold}, 命題
    \ref{prop:dim of C^r-submanifold}より, 
    次のことを証明すればよい.\\ 

    $q\in N$の逆像$f^{-1}(q)$に属する任意の点
    $p$に対し, $p$のまわりの座標近傍
    $(U;x_1,\cdots .x_m)$が存在して, 
    $$f^{-1}(q)\cap U
    =\{(x_1,\cdots x_m)\in U|
    x_{m-n+1}=\cdots =x_m=0\}$$
    が成り立つ. \\

    今, $f(p)=q$を満たす$p\in M$について, 常に
    rank$(Jf)_p=n$であるから, $(df)_p$は上への
    写像である. よって, 定理\ref{theo: projection theorem}
    より, $p$のまわりの座標近傍$(U;x_1,\cdots ,x_m)$
    と$q$($=f(p)$)のまわりの座標近傍
    $(V;y_1,\cdots ,y_n)$が存在して, $f|U:U\to V$
    は
    $$(y_1,\cdots ,y_n)=f(x_1,\cdots x_m)
    =(x_{m-n+1},\cdots ,x_m)$$
    と座標表示される. \\
    $(U;x_1,\cdots x_m)$は任意にとってきた
    $(V;y_1,\cdots ,y_n)$
    に応じて選べるから, $(V;y_1,\cdots ,y_n)$は
    点$q$で$y_1=\cdots =y_n=0$となるように
    とっておくと, 
    \begin{eqnarray*}
        f^{-1}(q)\cap U&=& \{p\in U|f(p)=q\}\\
        &=&\{(x_1,\cdots ,x_m)\in U|f(x_1,\cdots x_m)=(0,\cdots ,0)\}\\
        &=&\{(x_1,\cdots ,x_m)\in U|(x_{m-n+1},\cdots x_m)=(0,\cdots ,0)\}\\
        &=&\{(x_1,\cdots x_m)\in U|x_{m-n+1}=\cdots =x_m=0\}
    \end{eqnarray*}
    となり, 条件を満たす$U$が存在することがわかる. 
    これで定理\ref{theo:f^{-1}(q) C^r manifold}
    が証明できた. 
\end{proof}
\subsection{多様体の次元の具体的な計算}
\newpage
%
\section{おわりに}
\newpage
%参考文献
\begin{thebibliography}{99}
\bibitem{Matsumoto18} 松本幸夫, [第30版]多様体の基礎, 東京大学出版会, 2018.
\bibitem{b}【論文の場合】著者名,タイトル,雑誌名,巻・号,出版年度,頁.
\bibitem{c}【Webページの場合】 タイトル,ページ制作者(機関)等,URL: \url{http://www.shibaura-it.ac.jp/},最終アクセス日時: 2021/12/28 16:33.
\end{thebibliography}

\end{document}