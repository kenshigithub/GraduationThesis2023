% 総合研究論文サンプルファイル
%
% ・和文題名, 英文題名, 学籍番号, 姓, 名, 指導教員名, 職位 を記入する
%   (====== で囲まれた範囲).
% ・顔写真 (履歴書の余りなど) を用意し, 所定の位置に貼る.
%   デジタルデータ (JPG 等) があれば, それを使ってもよい (下記設定変更).
%
\documentclass[a4j,12pt]{jarticle}
\usepackage{amsmath}
\usepackage{amssymb}
\usepackage[dvipdfmx]{graphicx}%画像挿入
%\usepackage{theorem}%定義や定理の環境
\usepackage{url}
%必要なパッケージを追加
%定理環境の設定

\usepackage{amsthm}
\theoremstyle{definition}
\newtheorem{theorem}{定理}[section]
\newtheorem{definition}[theorem]{定義}
\newtheorem{lemma}[theorem]{補題}
\newtheorem{axiom}[theorem]{公理}
\newtheorem{proposition}[theorem]{命題}
\newtheorem{corollary}[theorem]{系}
\newtheorem{example}[theorem]{例}
\newtheorem*{theorem*}{定理}
\newtheorem*{definition*}{定義}
\newtheorem*{lemma*}{補題}
\newtheorem*{axiom*}{公理}
\newtheorem*{proposition*}{命題}
\newtheorem*{corollary*}{系}
\newtheorem*{example*}{例}
\renewcommand\proofname{\bf 証明}

%スタイルファイルを追加したい場合は以下のように書く
\usepackage{graphicx}
\usepackage{amsmath,amssymb}
%\usepackage{eclbkbox}
\usepackage{tikz}

%\newcommandでマクロを定義できる
\newcommand{\macrotest}{\LaTeX のマクロは便利}
\newcommand{\PD}[2]{\frac{\partial {#1}}{\partial {#2}}}%引数もとれる
\newcommand{\combination}[2]{{}_{#1} \mathrm{C}_{#2}}



%========================================
% 各自の内容を記述
%
\def\YEAR{2023}  % 総合研究着手年度
\def\TITLE{多様体の次元を調べる方法}  % 和文題名
\def\ETITLE{Methods for Investigating the Dimension of Manifolds}  % 英文題名
\def\NUMBER{BV20052}  % 学籍番号
\def\FAMILYNAME{\ruby{青見}{あおみ}}  % 姓
\def\FIRSTNAME{\ruby{健志}{けんし}}  % 名
\def\ADVISER{亀子 正樹}  % 指導教員 氏名
\def\POSITION{教授}  % 指導教員 職位
\def\JPG{SIT.png}  % 顔写真ファイル名 (なければ空白で可)
%
% 顔写真の設定: 下のいずれかの行を活かし, 他方をコメントアウト
%
\def\FACE{\WAKU}   % 枠線のみ表示
%\def\FACE{\IMAGE}  % 顔写真ファイル使用時
%========================================
% ルビの定義
%
\def\ruby#1#2{%
\leavevmode
\setbox0=\hbox{#1}\setbox1=\hbox{\scriptsize#2}%
\ifdim\wd0>\wd1 \dimen0=\wd0 \else \dimen0=\wd1 \fi
\hbox{\kanjiskip=\fill
 \vbox{\hbox to \dimen0{\scriptsize \hfil#2\hfil}%
 \nointerlineskip
 \hbox to \dimen0{\hfil#1\hfil}}}}
%
% 写真の定義
\def\WAKU{\framebox[30mm]{\rule{0mm}{38mm}\raisebox{18mm}{\smash{
 \parbox{25mm}{\begin{center}顔写真\\ 横3cm\\ ×\\ 縦4cm\end{center}}}}}
 \\[7mm]}
\def\IMAGE{\parbox{33mm}{\rule{0mm}{38mm}\raisebox{18mm}{\smash{
 \parbox{31mm}{\includegravarphics[height=40mm,keepaspectratio]{\JPG}}}}}
 \\[12mm]}
%----------------------------------------
\begin{document}
\thispagestyle{empty}
\begin{center}
 \YEAR{}年度 \\
 芝浦工業大学\quad システム理工学部 \\
 数理科学科 \\[12mm]
 \huge 総合研究論文
\end{center}
\vspace{16mm}\par
%
% 題名
%
\noindent\smash{
 \begin{minipage}[t]{.98\linewidth}
  \begin{center}
   \Huge\bf\TITLE  % 和文題名
   \\[1ex]
   \Large\bf\ETITLE  % 英文題名
  \end{center}
 \end{minipage}}
\vspace{47mm}\par
%
% 顔写真, 学籍番号, 氏名, 指導教員
%
\begin{center}
 \FACE                                        % 顔写真
 \Large \NUMBER                     \\[7mm]   % 学籍番号
 \huge  \FAMILYNAME\quad \FIRSTNAME \\[15mm]  % 氏名
 \Large 指導教員: \ADVISER ~ \POSITION
\end{center}
\vfill
\newpage
\setcounter{page}{1}
\tableofcontents	%目次
\thispagestyle{empty}
\newpage
%%%%%%↓ここから本文↓%%%%%%
\section{はじめに}
\newpage
%
\section{$C^r$級多様体と$C^r$級写像}
\begin{definition}
    位相空間$X$の開集合$U$から$m$次元数空間$\mathbb{R}^m$
    のある開集合$U'$への同相写像
    $$\varphi:U\rightarrow U'$$
    があるとき, $(U, \varphi)$を$m$次元座標近傍といい, 
    $\varphi$を$U$上の局所座標系という. 
\end{definition}
\begin{definition}
    $(U, \varphi)$を位相空間$X$内の局所座標近傍とする.
    $U$内の任意の点$p$に対して$\varphi(p) \in \mathbb{R}^m$
    であるから, 
    $$\varphi(p)=(x_1, \cdots ,x_m)$$
    と書ける.$(x_1, \cdots ,x_m)$を$(U, \varphi)$に関する$p$
    の局所座標という.
\end{definition}
\begin{definition}
    位相空間$M$が次の条件(1),(2)を満たすとき, 
    $M$を$m$次元位相多様体という. 
    \begin{itemize}
        \item[(1)]$M$はハウスドルフ空間である.
        \item[(2)]$M$内の任意の点$p$に対して, 
        $p$を含む$m$次元座標近傍$(U,\varphi)$が存在する.
    \end{itemize}
\end{definition}
\begin{definition}
    $m$次元位相多様体$M$の$2$つの座標近傍$(U, \varphi)$, 
    $(V, \psi)$が交わっているとき, 同相写像
    $$\psi \circ \varphi:\varphi(U\cap V)\rightarrow \psi(U\cap V)$$
    を$(U, \varphi)$から$(V, \psi)$への座標変換という. 
\end{definition}
\begin{definition}\label{def:C^r manifold}
    $r\geq 1$を自然数または$\infty$とする. 
    位相空間$M$が次の条件(1), (2), (3)を満たすとき, 
    $M$を$m$次元$C^r$級多様体という.
    \begin{itemize}
        \item[(1)]$M$はハウスドルフ空間である.
        \item[(2)]$M$は$m$次元座標近傍により被覆される. 
        すなわち, $M$の$m$次元座標近傍からなる族
        $\{(U_\alpha, \varphi_\alpha)\}_{\alpha \in A}$
        があって, 
        $$M = \bigcup_{\alpha \in A}U_\alpha$$
        が成り立つ. 
        \item[(3)]$U_\alpha \cap U_\beta \neq \phi$
        であるような任意の$\alpha$, $\beta$に対して, 座標変換
        $$\psi \circ \varphi:\varphi(U\cap V)\rightarrow \psi(U\cap V)$$
        は$C^r$級写像である. 
    \end{itemize}
\end{definition}
\begin{theorem}
    $m$次元球面$S^m \in \mathbb{R}^{m+1}$を
    $$S^m=\{(x_1,\cdots x_{m+1})|x_1^2+\cdots +x_{m+1}^2=1\}$$
    と定義すると, $S^m$は$m$次元$C^{\infty}$級多様体である. 
\end{theorem}
\begin{proof}
    定義\ref{def:C^r manifold}の条件(1), (2), (3)を確かめる. 
    \begin{itemize}
        \item[(1)]$\mathbb{R}^{m+1}$はハウスドルフ空間
        であるから, その部分空間として, $S^m$は
        ハウスドルフ空間である. 
        \item[(2)]$S^m$の$2(m+1)$個の開集合
        $U_i^+$, $U_i^-$ $(i=1,\cdots ,m+1)$を
        次のように定義する. 
        $$U_i^+ = \{(x_1, \cdots x_i, \cdots ,x_{m+1})\in S^m|x_i>0\}$$
        $$U_i^- = \{(x_1, \cdots x_i, \cdots ,x_{m+1})\in S^m|x_i<0\}$$
        $S^m$はこれら$U_i^+$, $U_i^-$ $(i=1,\cdots ,m+1)$
        で被覆される. 写像$\varphi_i^+:U_i^+ \rightarrow \mathbb{R}^m$, 
        $\varphi_i^-:U_i^- \rightarrow \mathbb{R}^m$を
        それぞれ次のように定義する. 
        $$\varphi_i^+(x_1,\cdots ,x_i,\cdots, x_{m+1})=(x_1,\cdots ,\hat{x_i},\cdots ,x_{m+1})$$
        $$\varphi_i^-(x_1,\cdots ,x_i,\cdots, x_{m+1})=(x_1,\cdots ,\hat{x_i},\cdots ,x_{m+1})$$
        ここで, $\hat{x_i}$は$x_i$を取り去るという意味である. このとき, 
        $$(\varphi_i^+)^{-1}(x_1,\cdots ,\hat{x_i},\cdots ,x_{m+1})=(x_1,\cdots ,\sqrt{1-||(x_1,\cdots ,\hat{x_i},\cdots ,x_{m+1})||^2},\cdots ,x_{m+1})$$
        $$(\varphi_i^-)^{-1}(x_1,\cdots ,\hat{x_i},\cdots ,x_{m+1})=(x_1,\cdots ,-\sqrt{1-||(x_1,\cdots ,\hat{x_i},\cdots ,x_{m+1})||^2},\cdots ,x_{m+1})$$
        であり, $\varphi_i^+$, $\varphi_i^-$はそれぞれ, $U_i^+$, 
        $U_i^-$から$\mathring{D}^m$への同相写像である.
        ただし, $\mathring{D}^m$は$\mathbb{R}^m$の原点を
        中心とするm次元単位開円板である. 
        よって, $S^m$は$2(m+1)$個の座標近傍$(U_1^+,\varphi_1^+),
        (U_1^-,\varphi_1^-),\cdots ,(U_{m+1}^+,\varphi_{m+1}^+),(U_{m+1}^-,\varphi_{m+1}^-)$
        で被覆される. 
        \item[(3)]2(m+1)個の座標近傍の間の座標変換がすべて
        $C^{\infty}$級であることを示す. \\
        $1\leq a, b\leq 2(m+1)$を満たす互いに異なる自然数$a$, $b$
        に対して, 
        \begin{itemize}
            \item[(i)]$(U_a^+,\varphi_a^+)$と$(U_b^+,\varphi_b^+)$
            \item[(ii)]$(U_a^-,\varphi_a^-)$と$(U_b^-,\varphi_b^-)$
            \item[(iii)] $(U_a^+,\varphi_a^+)$と$(U_b^-,\varphi_b^-)$ 
        \end{itemize}
        の間の座標変換を調べればよい. \\
        (i)の場合, 
        \begin{eqnarray*}
            &&U_a^+\cap U_b^+ =\{(x_1,\cdots ,x_a,\cdots ,x_b,\cdots ,x_{m+1})\in S^m|x_a>0,\ x_b>0\}\\
            &&\varphi_a^+(U_a^+\cap U_b^+) =\{(x_1,\cdots ,\hat{x_a},\cdots ,x_b,\cdots ,x_{m+1})\in \mathring{D}^m|x_b>0\}\\
            &&\varphi_b^+(U_a^+\cap U_b^+) =\{(x_1,\cdots ,x_a,\cdots ,\hat{x_b},\cdots ,x_{m+1})\in \mathring{D}^m|x_a>0\}\\
            &&(\varphi_a^+)^{-1}x_1,\cdots ,\hat{x_a},\cdots ,x_b,\cdots x_{m+1})\\
            &&=(x_1,\cdots ,\sqrt{1-||(x_1,\cdots ,\hat{x_a},\cdots ,x_b,\cdots x_{m+1})||^2},\cdots ,x_b,\cdots x_{m+1})
        \end{eqnarray*}
        この式から
        \begin{eqnarray*}
        &&\varphi_b^+\circ(\varphi_a^+)^{-1}(x_1,\cdots ,\hat{x_a},\cdots ,x_b,\cdots ,x_{m+1})\\
        &&=(x_1,\cdots ,\sqrt{1-||(x_1,\cdots ,\hat{x_a},\cdots ,x_b,\cdots ,x_{m+1})||^2},\cdots ,\hat{x_b},\cdots ,x_{m+1})
        \end{eqnarray*}
        これは$||(x_1,\cdots ,\hat{x_a},\cdots ,x_b,\cdots ,x_{m+1})||^2<1$の
        範囲で$C^{\infty}$級なので, \\$\varphi_b^+\circ(\varphi_a^+)^{-1}$
        は定義域$\varphi_a^+(U_a^+\cap U_b^+)\subset \mathring{D}^m$
        で$C^{\infty}$級である. \\
        同様にして(ii), (iii)の場合についても座標変換が$C^{\infty}$級である
        こと分かる. \\
    \end{itemize}
    以上より, $S^m$は$m$次元$C^{infty}$級多様体であることが分かった. 
\end{proof}
\subsection{小節のタイトル}
\subsubsection{小小節のタイトル}
\newpage
%
\section{おわりに}
\newpage
%参考文献
\begin{thebibliography}{99}
\bibitem{Matsumoto18} 松本幸夫, [第30版]多様体の基礎, 東京大学出版会, 2018.
\bibitem{b}【論文の場合】著者名,タイトル,雑誌名,巻・号,出版年度,頁.
\bibitem{c}【Webページの場合】 タイトル,ページ制作者(機関)等,URL: \url{http://www.shibaura-it.ac.jp/},最終アクセス日時: 2021/12/28 16:33.
\end{thebibliography}

\end{document}