% 総合研究論文サンプルファイル
%
% ・和文題名, 英文題名, 学籍番号, 姓, 名, 指導教員名, 職位 を記入する
%   (====== で囲まれた範囲).
% ・顔写真 (履歴書の余りなど) を用意し, 所定の位置に貼る.
%   デジタルデータ (JPG 等) があれば, それを使ってもよい (下記設定変更).
%
\documentclass[a4j,12pt]{jarticle}
\usepackage{amsmath}
\usepackage{amssymb}
\usepackage[dvipdfmx]{graphicx}%画像挿入
%\usepackage{theorem}%定義や定理の環境
\usepackage{url}
%必要なパッケージを追加
%定理環境の設定

\usepackage{amsthm}
\theoremstyle{definition}
\newtheorem{theorem}{定理}[section]
\newtheorem{definition}[theorem]{定義}
\newtheorem{lemma}[theorem]{補題}
\newtheorem{axiom}[theorem]{公理}
\newtheorem{proposition}[theorem]{命題}
\newtheorem{corollary}[theorem]{系}
\newtheorem{example}[theorem]{例}
\newtheorem*{theorem*}{定理}
\newtheorem*{definition*}{定義}
\newtheorem*{lemma*}{補題}
\newtheorem*{axiom*}{公理}
\newtheorem*{proposition*}{命題}
\newtheorem*{corollary*}{系}
\newtheorem*{example*}{例}
\renewcommand\proofname{\bf 証明}

%スタイルファイルを追加したい場合は以下のように書く
\usepackage{graphicx}
\usepackage{amsmath,amssymb}
%\usepackage{eclbkbox}
\usepackage{tikz}

%\newcommandでマクロを定義できる
\newcommand{\macrotest}{\LaTeX のマクロは便利}
\newcommand{\PD}[2]{\frac{\partial {#1}}{\partial {#2}}}%引数もとれる
\newcommand{\combination}[2]{{}_{#1} \mathrm{C}_{#2}}



%========================================
% 各自の内容を記述
%
\def\YEAR{2023}  % 総合研究着手年度
\def\TITLE{多様体の次元を調べる方法}  % 和文題名
\def\ETITLE{Methods for Investigating the Dimension of Manifolds}  % 英文題名
\def\NUMBER{BV20052}  % 学籍番号
\def\FAMILYNAME{\ruby{青見}{あおみ}}  % 姓
\def\FIRSTNAME{\ruby{健志}{けんし}}  % 名
\def\ADVISER{亀子 正樹}  % 指導教員 氏名
\def\POSITION{教授}  % 指導教員 職位
\def\JPG{SIT.png}  % 顔写真ファイル名 (なければ空白で可)
%
% 顔写真の設定: 下のいずれかの行を活かし, 他方をコメントアウト
%
\def\FACE{\WAKU}   % 枠線のみ表示
%\def\FACE{\IMAGE}  % 顔写真ファイル使用時
%========================================
% ルビの定義
%
\def\ruby#1#2{%
\leavevmode
\setbox0=\hbox{#1}\setbox1=\hbox{\scriptsize#2}%
\ifdim\wd0>\wd1 \dimen0=\wd0 \else \dimen0=\wd1 \fi
\hbox{\kanjiskip=\fill
 \vbox{\hbox to \dimen0{\scriptsize \hfil#2\hfil}%
 \nointerlineskip
 \hbox to \dimen0{\hfil#1\hfil}}}}
%
% 写真の定義
\def\WAKU{\framebox[30mm]{\rule{0mm}{38mm}\raisebox{18mm}{\smash{
 \parbox{25mm}{\begin{center}顔写真\\ 横3cm\\ ×\\ 縦4cm\end{center}}}}}
 \\[7mm]}
\def\IMAGE{\parbox{33mm}{\rule{0mm}{38mm}\raisebox{18mm}{\smash{
 \parbox{31mm}{\includegravarphics[height=40mm,keepaspectratio]{\JPG}}}}}
 \\[12mm]}
%----------------------------------------
\begin{document}
\thispagestyle{empty}
\begin{center}
 \YEAR{}年度 \\
 芝浦工業大学\quad システム理工学部 \\
 数理科学科 \\[12mm]
 \huge 総合研究論文
\end{center}
\vspace{16mm}\par
%
% 題名
%
\noindent\smash{
 \begin{minipage}[t]{.98\linewidth}
  \begin{center}
   \Huge\bf\TITLE  % 和文題名
   \\[1ex]
   \Large\bf\ETITLE  % 英文題名
  \end{center}
 \end{minipage}}
\vspace{47mm}\par
%
% 顔写真, 学籍番号, 氏名, 指導教員
%
\begin{center}
 \FACE                                        % 顔写真
 \Large \NUMBER                     \\[7mm]   % 学籍番号
 \huge  \FAMILYNAME\quad \FIRSTNAME \\[15mm]  % 氏名
 \Large 指導教員: \ADVISER ~ \POSITION
\end{center}
\vfill
\newpage
\setcounter{page}{1}
\tableofcontents	%目次
\thispagestyle{empty}
\newpage
%%%%%%↓ここから本文↓%%%%%%
\section{はじめに}
\newpage
%
\section{準備}
%\subsection{m次元数空間}
%%\subsection{ベクトル空間}
\subsection{連続写像と$C^r$級写像}
\subsection{位相空間}
\newpage
%
\section{$C^r$級多様体と$C^r$級写像}
\begin{definition}
    位相空間$X$の開集合$U$から$m$次元数空間$\mathbb{R}^m$
    のある開集合$U'$への同相写像
    $$\varphi:U\rightarrow U'$$
    があるとき, $(U, \varphi)$を$m$次元座標近傍といい, 
    $\varphi$を$U$上の局所座標系という. 
\end{definition}
\begin{definition}
    $(U, \varphi)$を位相空間$X$内の局所座標近傍とする.
    $U$内の任意の点$p$に対して$\varphi(p) \in \mathbb{R}^m$
    であるから, 
    $$\varphi(p)=(x_1, \cdots ,x_m)$$
    と書ける.$(x_1, \cdots ,x_m)$を$(U, \varphi)$に関する$p$
    の局所座標という.
\end{definition}
\begin{definition}
    位相空間$M$が次の条件(1),(2)を満たすとき, 
    $M$を$m$次元位相多様体という. 
    \begin{itemize}
        \item[(1)]$M$はハウスドルフ空間である.
        \item[(2)]$M$内の任意の点$p$に対して, 
        $p$を含む$m$次元座標近傍$(U,\varphi)$が存在する.
    \end{itemize}
\end{definition}
\begin{definition}
    $m$次元位相多様体$M$の$2$つの座標近傍$(U, \varphi)$, 
    $(V, \psi)$が交わっているとき, 同相写像
    $$\psi \circ \varphi:\varphi(U\cap V)\rightarrow \psi(U\cap V)$$
    を$(U, \varphi)$から$(V, \psi)$への座標変換という. 
\end{definition}
\begin{definition}\label{def:C^r manifold}
    $r\geq 1$を自然数または$\infty$とする. 
    位相空間$M$が次の条件(1), (2), (3)を満たすとき, 
    $M$を$m$次元$C^r$級多様体という.
    \begin{itemize}
        \item[(1)]$M$はハウスドルフ空間である.
        \item[(2)]$M$は$m$次元座標近傍により被覆される. 
        すなわち, $M$の$m$次元座標近傍からなる族
        $\{(U_\alpha, \varphi_\alpha)\}_{\alpha \in A}$
        があって, 
        $$M = \bigcup_{\alpha \in A}U_\alpha$$
        が成り立つ. 
        \item[(3)]$U_\alpha \cap U_\beta \neq \phi$
        であるような任意の$\alpha$, $\beta$に対して, 座標変換
        $$\psi \circ \varphi:\varphi(U\cap V)\rightarrow \psi(U\cap V)$$
        は$C^r$級写像である. 
    \end{itemize}
\end{definition}
\begin{theorem}
    $m$次元球面$S^m \in \mathbb{R}^{m+1}$を
    $$S^m=\{(x_1,\cdots x_{m+1})|x_1^2+\cdots +x_{m+1}^2=1\}$$
    と定義すると, $S^m$は$m$次元$C^{\infty}$級多様体である. 
\end{theorem}
\begin{proof}
    定義\ref{def:C^r manifold}の条件(1), (2), (3)を確かめる. 
    \begin{itemize}
        \item[(1)]$\mathbb{R}^{m+1}$はハウスドルフ空間
        であるから, その部分空間として, $S^m$は
        ハウスドルフ空間である. 
        \item[(2)]$S^m$の$2(m+1)$個の開集合
        $U_i^+$, $U_i^-$ $(i=1,\cdots ,m+1)$を
        次のように定義する. 
        $$U_i^+ = \{(x_1, \cdots x_i, \cdots ,x_{m+1})\in S^m|x_i>0\}$$
        $$U_i^- = \{(x_1, \cdots x_i, \cdots ,x_{m+1})\in S^m|x_i<0\}$$
        $S^m$はこれら$U_i^+$, $U_i^-$ $(i=1,\cdots ,m+1)$
        で被覆される. 写像$\varphi_i^+:U_i^+ \rightarrow \mathbb{R}^m$, 
        $\varphi_i^-:U_i^- \rightarrow \mathbb{R}^m$を
        それぞれ次のように定義する. 
        $$\varphi_i^+(x_1,\cdots ,x_i,\cdots, x_{m+1})=(x_1,\cdots ,\hat{x_i},\cdots ,x_{m+1})$$
        $$\varphi_i^-(x_1,\cdots ,x_i,\cdots, x_{m+1})=(x_1,\cdots ,\hat{x_i},\cdots ,x_{m+1})$$
        ここで, $\hat{x_i}$は$x_i$を取り去るという意味である. このとき, 
        $$(\varphi_i^+)^{-1}(x_1,\cdots ,\hat{x_i},\cdots ,x_{m+1})=(x_1,\cdots ,\sqrt{1-||(x_1,\cdots ,\hat{x_i},\cdots ,x_{m+1})||^2},\cdots ,x_{m+1})$$
        $$(\varphi_i^-)^{-1}(x_1,\cdots ,\hat{x_i},\cdots ,x_{m+1})=(x_1,\cdots ,-\sqrt{1-||(x_1,\cdots ,\hat{x_i},\cdots ,x_{m+1})||^2},\cdots ,x_{m+1})$$
        であり, $\varphi_i^+$, $\varphi_i^-$はそれぞれ, $U_i^+$, 
        $U_i^-$から$\mathring{D}^m$への同相写像である.
        ただし, $\mathring{D}^m$は$\mathbb{R}^m$の原点を
        中心とするm次元単位開円板である. 
        よって, $S^m$は$2(m+1)$個の座標近傍$(U_1^+,\varphi_1^+),
        (U_1^-,\varphi_1^-),\cdots ,(U_{m+1}^+,\varphi_{m+1}^+),(U_{m+1}^-,\varphi_{m+1}^-)$
        で被覆される. 
        \item[(3)]2(m+1)個の座標近傍の間の座標変換がすべて
        $C^{\infty}$級であることを示す. \\
        $1\leq a, b\leq 2(m+1)$を満たす互いに異なる自然数$a$, $b$
        に対して, 
        \begin{itemize}
            \item[(i)]$(U_a^+,\varphi_a^+)$と$(U_b^+,\varphi_b^+)$
            \item[(ii)]$(U_a^-,\varphi_a^-)$と$(U_b^-,\varphi_b^-)$
            \item[(iii)] $(U_a^+,\varphi_a^+)$と$(U_b^-,\varphi_b^-)$ 
        \end{itemize}
        の間の座標変換を調べればよい. \\
        (i)の場合, 
        \begin{eqnarray*}
            &&U_a^+\cap U_b^+ =\{(x_1,\cdots ,x_a,\cdots ,x_b,\cdots ,x_{m+1})\in S^m|x_a>0,\ x_b>0\}\\
            &&\varphi_a^+(U_a^+\cap U_b^+) =\{(x_1,\cdots ,\hat{x_a},\cdots ,x_b,\cdots ,x_{m+1})\in \mathring{D}^m|x_b>0\}\\
            &&\varphi_b^+(U_a^+\cap U_b^+) =\{(x_1,\cdots ,x_a,\cdots ,\hat{x_b},\cdots ,x_{m+1})\in \mathring{D}^m|x_a>0\}\\
            &&(\varphi_a^+)^{-1}x_1,\cdots ,\hat{x_a},\cdots ,x_b,\cdots x_{m+1})\\
            &&=(x_1,\cdots ,\sqrt{1-||(x_1,\cdots ,\hat{x_a},\cdots ,x_b,\cdots x_{m+1})||^2},\cdots ,x_b,\cdots x_{m+1})
        \end{eqnarray*}
        この式から
        \begin{eqnarray*}
        &&\varphi_b^+\circ(\varphi_a^+)^{-1}(x_1,\cdots ,\hat{x_a},\cdots ,x_b,\cdots ,x_{m+1})\\
        &&=(x_1,\cdots ,\sqrt{1-||(x_1,\cdots ,\hat{x_a},\cdots ,x_b,\cdots ,x_{m+1})||^2},\cdots ,\hat{x_b},\cdots ,x_{m+1})
        \end{eqnarray*}
        これは$||(x_1,\cdots ,\hat{x_a},\cdots ,x_b,\cdots ,x_{m+1})||^2<1$の
        範囲で$C^{\infty}$級なので, \\$\varphi_b^+\circ(\varphi_a^+)^{-1}$
        は定義域$\varphi_a^+(U_a^+\cap U_b^+)\subset \mathring{D}^m$
        で$C^{\infty}$級である. \\
        同様にして(ii), (iii)の場合についても座標変換が$C^{\infty}$級である
        こと分かる. \\
    \end{itemize}
    以上より, $S^m$は$m$次元$C^{\infty}$級多様体であることが分かった. 
\end{proof}
\begin{proposition}\label{prop: cord-nabor condition}
    $M$を$m$次元$C^r$級多様体, $V$を$M$の開集合, 
    $V'$を$\mathbb{R}^m$の開集合とする. また, 
    $\varphi:V\to V'$を同相写像とする. 
    このとき, $(V,\varphi)$が$M$の$C^r$級
    座標近傍になるための必要十分条件は, 
    $\varphi:V\to V'$が, $C^r$級微分同相写像
    であることである. 
\end{proposition}

%
\section{接ベクトル空間}
\subsection{接ベクトル空間}
\subsection{$C^r$級写像の微分}
\newpage
%
\section{多様体の次元を調べる方法}
\subsection{写像の局所的性質}
\begin{theorem}\label{theo:f^(-1)theorem}
    $(df)_p:T_p(M)\to T_p(N)$が線形写像として
    同型なら, $f$は$p$のある開近傍から$f(p)$の
    ある開近傍への$C^r$級微分同相写像である. 
    すなわち, $p$の開近傍$U$と$f(p)$の開近傍$V$
    が存在して, $f(U)=V$となり, かつ, 
    $f|U:U \to V$は$C^r$級微分同相写像である. 
\end{theorem}
\begin{theorem}\label{theo: projection theorem}
    $f:M\to N$を$C^r$級写像とする. ある点$p\in M$
    における微分$(df)_p:T_p(M)\to T_p(N)$が上への
    線形写像なら, 点$p$付近での$f$の様子は, 射影:
    $\mathbb{R}^{m-n} \times \mathbb{R}^n \to \mathbb{R}^n$, 
    $(x_1, \cdots ,x_m)\mapsto (x_(m-n+1), \cdots ,x_m)$
    と同じである. すなわち, $p$のまわりの局所座標系
    $(x_1,\cdots ,x_m)$と$f(p)$のまわりの局所座標系
    $(y_1, \cdots ,y_n)$をうまく選んで, $f$
    の局所座標表示
    $(y_1, \cdots ,y_n)=(c)$が
    \begin{eqnarray*}
        y_1&=&f_1(x_1,\cdots ,x_m)=x_{m-n+1}\\
        &\vdots& \\
        y_n&=&f_n(x_1,\cdots ,x_m)=x_m
    \end{eqnarray*}
    であるようにできる. 
\end{theorem}
\begin{proof}
    $C^r$級写像$f:M\to N$が与えられており, 
    ある点$p\in M$における微分
    $(df)_p:T_p(M)\to T_p(N)$が上への線形写像
    であるとする. このとき, rank$(df)_p=n$
    (ただし, $m=$dim$M\geq n=$dim$N$. )
    点$p$, 点$f(p)$のまわりに, それぞれ座標近傍
    $(U;x_1,\cdots ,x_m)$, $(V;y_1,\cdots ,y_n)$
    をとり, $f$を局所座標表示すると
    $$(y_1, \cdots ,y_n)=(f_1(x_1,\cdots ,x_m), 
    \cdots ,f_n(x_1,\cdots ,x_m))$$
    ヤコビ行列$(Jf)_p$は$(df)_p$を表す行列で, 
    $n$行$m$列である. rank$(df)_p=n$であるから, 
    $(Jf)_p$から$n$本の列ベクトルを選んで, 作った
    正方行列は正則である. 必要なら適当に列を入れ替えて
    $$B=
    \begin{pmatrix}
        \frac{\partial f_1}{\partial x_{m-n+1}}(p) & \cdots & \frac{\partial f_1}{\partial x_m}(p)  \\
        &\cdots& \\
        \frac{\partial f_n}{\partial x_{m-n+1}}(p) & \cdots & \frac{\partial f_n}{\partial x_m}(p)  \\
    \end{pmatrix}
    $$
    とおくと, det$B \neq 0$であって, 
    $$(Jf)_p=
    \left(
        \begin{array}{ccc|c} 
          * & \cdots & *  &  \\
           &  &   & B \\
          * & \cdots & *  & 
        \end{array} 
        \right)
$$
と仮定してよい. $p$のまわりの座標近傍
$(U;x_1,\cdots ,x_m)$から, $\mathbb{R}^m$への
写像$\varphi :U\to \mathbb{R}^m$を
$$\varphi(x_1,\cdots ,x_m)=(x_1, \cdots ,x_{m-n},f_1(x_1,\cdots ,x_m), 
\cdots ,f_n(x_1,\cdots ,x_m))$$
と定義する. やコビ行列$(J\varphi)_p$は
$$(J\varphi)_p=
    \left(
        \begin{array}{ccc|ccc} 
          1 &        &   & & & \\
            & \ddots &   & &O& \\
            &        & 1 & & &  \\ \hline
          * & \cdots & * & & &  \\
            & \cdots &   & &B& \\
          * & \cdots & * & & &
        \end{array} 
        \right)
$$
となり, det$(J\varphi)_p=$det$B\neq 0$である. 
したがって, 逆関数の定理(定理
\ref{theo:f^(-1) theorem})
により, 点$p$を含む$U$を十分小さくとれば, 
$\varphi |U:U\to \varphi(U)$は$C^r$級微分同相写像
である. よって, /ref{prop: cord-nabor condition}
より,$(U, \varphi|U)$は$p$のまわりの新しい$C^r$
級座標近傍と思える. $(U, \varphi|U)$の
局所座標系を$(z_1,\cdots ,z_m)$とすると, 
上の$\varphi$の定義式から, 
$$(z_1,\cdots ,z_m)=\varphi(x_1,\cdots x_m)
=(x_1,\cdots x_{m-n},f_1(x_1,\cdots ,x_m), 
\cdots ,f_n(x_1,\cdots ,x_m))$$
を得る. とくに, $(z_{m-n+1},\cdots ,z_m)
=(f_1(x_1,\cdots ,x_m), 
\cdots ,f_n(x_1,\cdots ,x_m))$
であるから, 
\begin{eqnarray*}
    f\circ \varphi (z_1,\cdots z_m)&=&f(x_1,\cdots x_m)\\
    &=&(f_1(x_1,\cdots ,x_m),\cdots ,f_n(x_1,\cdots ,x_m))\\
    &=&(z_{m-n+1},\cdots ,z_m)
\end{eqnarray*}
となる. よって, $(z_1,\cdots ,z_m)$と$(y_1,\cdots ,y_n)$
に関する$f:M\to N$の点$p$のまわりでの局所座標表示
$$(z_1,\cdots ,z_m)\mapsto (z_{m-n+1},\cdots ,z_m)
=(y_1,\cdots ,y_n)$$
である. $(z_1,\cdots ,z_m)$を改めて$(x_1,\cdots ,x_m)$
と書き直せば, 定理\ref{theo: projection theorem}の
主張が得られる. 
\end{proof}
\begin{theorem}\label{theo:inclusion map theorem}
    $f:M\to N$を$C^r$級写像とする. 
    $(df)_p:T_p(M)\to T_{f(p)}(N)$が
    $1$対$1$の線形写像なら, 点$p$の付近での$f$の
    様子は, 包含写像
    $\mathbb{R}^m\to \mathbb{R}^n,\ 
    (x_1,\cdots ,x_m)\mapsto 
    (x_1,\cdots ,x_m,0,\cdots ,0)$
    と同じである. すなわち, $p$のまわりの局所座標系
    $(x_1,\cdots ,x_m)$と$f(p)$のまわりの局所座標系
    $(y_1,\cdots ,y_n)$をうまく選んで, $f$の
    局所座標表示
    $(y_1,\cdots ,y_n)\mapsto (f_1(x_1,\cdots x_m),
    \cdots ,f_n(x_1,\cdots x_m),)$が, 
    \begin{eqnarray*}
        y_1&=&f_1(x_1,\cdots x_m)=x_1\\
        &\vdots& \\
        y_m&=&f_m(x_1,\cdots x_m)=x_m\\
        y_{m+1}&=&f_{m+1}(x_1,\cdots x_m)=0\\
        &\vdots& \\
        y_n&=&f_n(x_1,\cdots x_m)=0\\
    \end{eqnarray*}
    であるようにできる. 
\end{theorem}
\begin{proof}
    問題は局所的であるから$M=\mathbb{R}^m$, 
    $N=\mathbb{R}^n$と仮定してよい. ($m\leq n$)
    また, $p=\boldsymbol{o}$, $f(p)=\boldsymbol{o}$
    としてよい. 
    $\mathbb{R}^m$の自然な座標$(x_1, \cdots ,x_m)$
    と, $\mathbb{R}^n$の自然な座標に関して, 
    与えられた$C^r$級写像$f:\mathbb{R}^m\to \mathbb{R}^n$
    を局所座標表示したものを
    $$u_1=f_1(x_1,\cdots ,x_m),\ \cdots ,
    u_n=f_n(x_1,\cdots ,x_m)$$
    とする. $(df)_{\boldsymbol{o}}:
    T_{\boldsymbol{o}}(\mathbb{R}^m)\to 
    T_{\boldsymbol{o}}(\mathbb{R}^n)$
    は$1$対$1$であるから, ヤコビ行列$(Jf)_{\boldsymbol{o}}$
    から, 適当な$m$行を選び出して作った$m$次正方行列
    は正則である. 必要なら, 
    $(u_1, \cdots ,u_m)$の並び方を変えて, 
    $(Jf)_{\boldsymbol{o}}=
    \begin{pmatrix}
        A \\ \hline
        B
     \end{pmatrix}
     $
     ($A$は$m$行$m$列の正則行列)と仮定してよい. 
     $(x_1, \cdots ,x_m)$に新しく, 
     $(x_{m+1}, \cdots ,x_n)$を付け加えて, 
     $\mathbb{R}^m \times \mathbb{R}^{n-m}$を
     構成し, 新しい写像
     $F:\mathbb{R}^m \times \mathbb{R}^{n-m}
     \to \mathbb{R}^n$
    を, 
    \begin{eqnarray*}
        &&F(x_1,\cdots ,x_m,x_{m+1},\cdots x_n)\\
    &&=(f_1(x_1,\cdots ,x_m),\cdots ,f_m(x_1,\cdots ,x_m), 
    f_{m+1}(x_1,\cdots ,x_m)+x_{m+1}, 
    \cdots ,f_{n}(x_1,\cdots ,x_m)+x_{n})
    \end{eqnarray*}
    と定義する.$F$のヤコビ行列
    $$(Jf)_{\boldsymbol{o}}=
    \left(
\begin{array}{c|ccc} 
  A &  & O  &  \\ \hline
   & 1 &   &  \\
  B &  & \vdots &  \\
   &  &   & 1
\end{array} 
\right)
    $$
    であり, 正則である. 
    逆関数の定理\ref{theo:f^(-1)theorem}により, 
    $\mathbb{R}^m \times \mathbb{R}^{n-m}$
    における, $\boldsymbol{o}$の近傍$U$と, 
    $\mathbb{R}^n$における$\boldsymbol{o}$
    の近傍$V$が存在して, $F|U:U\to V$は$C^r$級
    微分同相写像になる. $\psi =(F|U)^{-1}:V\to U$
    とおく. $(V, \psi)$は$\boldsymbol{o}$の
    まわりの$\mathbb{R}^n$の$C^r$級座標近傍と思える. 
    この座標近傍に関する局所座標系を$(y_1,\cdots y_n)$
    とし, $(x_1, \cdots ,x_m)$と$(y_1,\cdots y_n)$に
    関して, $f:U\cap(\mathbb{R}^m\times 
    \{ \boldsymbol{o} \})\to \mathbb{R}^n$
    を局所座標表示すると, 
    \begin{eqnarray*}
        (y_1,\cdots ,y_n)&=&\psi (f(x_1,\cdots ,x_m))\\
        &=&\psi(F(x_1,\cdots ,x_m,0,\cdots 0))\\
        &=&(F|U)^{-1}(F(x_1,\cdots ,x_m,0,\cdots 0))\\
        &=&(x_1,\cdots ,x_m,0,\cdots ,0)
    \end{eqnarray*}
    となる. したがって, 定理\ref{theo:inclusion map theorem}
    の主張する通りの局所座標表示が得られた. 
\end{proof}
\subsection{$C^r$級部分多様体}
\begin{definition}\label{def:C^r-submanifold}
    $n$次元$C^r$級多様体$N$の部分集合$L$が
    $N$の$l$次元$C^r$級部分多様体であるとは, 
    \begin{itemize}
        \item[(1)]$l=n$のとき:$L$が$N$の開集合
        であることである. 
        \item[(2)] $0\leq l<n$のとき:$L$の任意の点$p$
        に対し, $p$を含む$N$の座標近傍$(U;x_1,\cdots ,x_n)$
        が存在して, 
        $$L\cap N=\{(x_1,\cdots ,x_n)\in U|
        x_{l+1}=\cdots =x_n=0\}$$
        が成り立つことである. 
    \end{itemize}
\end{definition}
\begin{proposition}\label{prop:dim of C^r-submanifold}
    $n$次元$C^r$級多様体$N$の$l$次元$C^r$級
    部分多様体$L$は, それ自身$l$次元$C^r$級
    多様体である. 
\end{proposition}
\begin{proof}
    \begin{itemize}
        \item[(1)]
        $l=n$のとき, $N$からの相対位相によって, 
        $L$は位相空間となり, 
        $N$の$C^r$級座標近傍を
        $\mathcal{S}=
        \{U_\alpha,\varphi _\alpha\}_{\alpha\in A}$
        とすると, $\mathcal{S}$を$L$に制限した
        $\{U_\alpha \cap L,
        \varphi _\alpha |U_\alpha \cap L\}_{\alpha\in A}$
        は$L$の$C^r$局所座標系になる. 
        よって, $L$はそれ自身, $l$($=n$)次元
        $C^r$級多様体である. 
        \item[(2)] 
        $0\leq l<n$のとき, $L$には$N$からの相対位相
        を入れる. $N$がハウスドルフ空間であるから$L$
        もそうである. \\
        $L$の任意の点$p$に対し, $p$を含む$N$の
        局所座標系$(U;x_1,\cdots ,x_n)$で定義
        \ref{def:C^r-submanifold}の条件を満たすもの
        を選び, $(U_p;x^p_1,\cdots ,x^p_n)$とする. 
        $V_p=L\cap U_p$とおくと, $V_p$は$L$の開集合
        である. $V_p$上の$U_p$の局所座標系
        $(x^p_1, \cdots ,x^p_n)$の$x^p_1$から
        $x^p_l$までを制限したもの$(V_p;x^p_1,\cdots ,x^p_l)$
        を考える. \\
        $\{(V_p;x^p_1,\cdots ,x^p_l)\}_{p\in L}$が
        $L$を被覆することは明らかである.\\
        $V_p$と$V_q$が交わるとする. 対応する
        $(U_p;x^p_1,\cdots ,x^p_n)$ と
        $(U_q;x^q_1,\cdots ,x^q_n)$は, 
        $N$の適当な座標近傍
        $(U_\alpha ;x^\alpha_1, \cdots ,x^\alpha_n)$,
        $(U_\beta ;x^\beta_1, \cdots ,x^\beta_n)$
        である. この間の座標変換はある$C^r$級関数$f$を
        用いて, 
        $$(x^\beta_1, \cdots ,x^\beta_n)
        =(f_1(x^\alpha_1, \cdots ,x^\alpha_n), \cdots 
        , f_n(x^\alpha_1, \cdots ,x^\alpha_n))$$
        と書ける. $V_p\cap V_q$上では, 
        $x^\alpha_{l+1}=x^\alpha_n=0$, $x^\beta_{l+1}=x^\beta_n=0$
        が成り立つので, $V_p\cap V_q$上では
        \begin{eqnarray*}
            x^\beta_1&=&f_1(x^\alpha_1, \cdots ,x^\alpha_l,0,\cdots ,0)\\
            &\vdots& \\
            x^\beta_l&=&f_l(x^\alpha_1, \cdots ,x^\alpha_l,0,\cdots ,0)\\
            0&=&f_{l+1}(x^\alpha_1, \cdots ,x^\alpha_l,0,\cdots ,0)\\
            &\vdots& \\
            0&=&f_n(x^\alpha_1, \cdots ,x^\alpha_l,0,\cdots ,0)\\
        \end{eqnarray*}
        となっている. 改めて関数$g$を
        \begin{eqnarray*}
            g(x^\alpha_1, \cdots ,x^\alpha_l)&=&
            (g_1(x^\alpha_1, \cdots ,x^\alpha_l), \cdots 
            , g_l(x^\alpha_1, \cdots ,x^\alpha_l))\\
            &=&(f_1(x^\alpha_1, \cdots ,x^\alpha_l,0,\cdots ,0), \cdots 
            , f_l(x^\alpha_1, \cdots ,x^\alpha_l,0,\cdots ,0))
        \end{eqnarray*}
        と定義すると, この関数は$C^r$級である. 
        そして, 
        $$(x^\beta_1, \cdots ,x^\beta_l)
        =g(x^\alpha_1, \cdots ,x^\alpha_l)$$
        が$(V_p;x^p_1,\cdots x^p_l)$から
        $(V_q;x^q_1,\cdots x^q_l)$の座標変換を与えている. \\
        ゆえに, $\{(V_p;x^p_1,\cdots ,x^p_l)\}_{p\in L}$
        は$L$の$C^r$級座標近傍になっている. \\
        以上より, $L$は$l$次元$C^r$級多様体である. 
    \end{itemize}
\end{proof}
\begin{theorem}\label{theo:f^{-1}(q) C^r manifold}
    $M$, $N$を$m$次元, $n$次元の$C^r$級多様体, 
    $f:M\to N$を$C^r$級写像とする. $N$のある点
    $q$について, $f(p)=q$となる$M$の各点$p$
    が常にrank$(Jf)_p=n$を満たすとき, 逆像
    $f^{-1}(q)$は$(m-n)$次元$C^r$級多様体
    である. 
\end{theorem}
\begin{proof}
    定義\ref{def:C^r-submanifold}, 命題
    \ref{prop:dim of C^r-submanifold}より, 
    次のことを証明すればよい.\\ 

    $q\in N$の逆像$f^{-1}(q)$に属する任意の点
    $p$に対し, $p$のまわりの座標近傍
    $(U;x_1,\cdots .x_m)$が存在して, 
    $$f^{-1}(q)\cap U
    =\{(x_1,\cdots x_m)\in U|
    x_{m-n+1}=\cdots =x_m=0\}$$
    が成り立つ. \\

    今, $f(p)=q$を満たす$p\in M$について, 常に
    rank$(Jf)_p=n$であるから, $(df)_p$は上への
    写像である. よって, 定理\ref{theo: projection theorem}
    より, $p$のまわりの座標近傍$(U;x_1,\cdots ,x_m)$
    と$q$($=f(p)$)のまわりの座標近傍
    $(V;y_1,\cdots ,y_n)$が存在して, $f|U:U\to V$
    は
    $$(y_1,\cdots ,y_n)=f(x_1,\cdots x_m)
    =(x_{m-n+1},\cdots ,x_m)$$
    と座標表示される. \\
    $(U;x_1,\cdots x_m)$は任意にとってきた
    $(V;y_1,\cdots ,y_n)$
    に応じて選べるから, $(V;y_1,\cdots ,y_n)$は
    点$q$で$y_1=\cdots =y_n=0$となるように
    とっておくと, 
    \begin{eqnarray*}
        f^{-1}(q)\cap U&=& \{p\in U|f(p)=q\}\\
        &=&\{(x_1,\cdots ,x_m)\in U|f(x_1,\cdots x_m)=(0,\cdots ,0)\}\\
        &=&\{(x_1,\cdots ,x_m)\in U|(x_{m-n+1},\cdots x_m)=(0,\cdots ,0)\}\\
        &=&\{(x_1,\cdots x_m)\in U|x_{m-n+1}=\cdots =x_m=0\}
    \end{eqnarray*}
    となり, 条件を満たす$U$が存在することがわかる. 
    これで定理\ref{theo:f^{-1}(q) C^r manifold}
    が証明できた. 
\end{proof}
\subsection{多様体の次元の具体的な計算}
\newpage
%
\section{おわりに}
\newpage
%参考文献
\begin{thebibliography}{99}
\bibitem{Matsumoto18} 松本幸夫, [第30版]多様体の基礎, 東京大学出版会, 2018.
\bibitem{b}【論文の場合】著者名,タイトル,雑誌名,巻・号,出版年度,頁.
\bibitem{c}【Webページの場合】 タイトル,ページ制作者(機関)等,URL: \url{http://www.shibaura-it.ac.jp/},最終アクセス日時: 2021/12/28 16:33.
\end{thebibliography}

\end{document}